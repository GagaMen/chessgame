% !TeX encoding=utf8
% !TeX spellcheck = de_DE
% !TeX root = ../Diploma.tex

\chapter{Fazit}\label{chap:conclusion}
Zielstellung dieser Arbeit war es herauszufinden, ob es möglich ist mit der Programmiersprache Kotlin sowohl den Server als auch den Client, am Beispiel eines Schachspiels, zu programmieren. Dabei wurde durch diese Arbeit gezeigt, dass dies keinerlei Probleme darstellt. Dennoch stellt sich die Frage, ob sich der Einsatz von Kotlin, als Ersatz für Java und JavaScript, lohnt. Zur Beantwortung dieser Frage wird in den \hyperref[sec:conclusionKotlinJava, sec:conclusionKotlinJS]{Abschnitten~\ref{sec:conclusionKotlinJava} und \ref{sec:conclusionKotlinJS}} jeweils für beide Sprachen ein Fazit gezogen.

\section{Verwendung von Kotlin für serverseitige Programmierung als Ersatz für Java}\label{sec:conclusionKotlinJava}
Grundlegend lässt sich sagen, dass die Programmierung mit Kotlin als Ersatz für Java sehr angenehm ist. Im Vergleich zu Kotlin wirkt Java teilweise sehr aufgebläht, was möglicherweise durch die benötigten \code{getter}- und \code{setter}-Methoden geschuldet ist. In Kotlin werden diese automatisch anhand der Klasseneigenschaften generiert, können aber bei Bedarf überschrieben werden\footnote{siehe \hyperref[lst:dbConnection]{Listing~\ref{lst:dbConnection}}}.\\
\\
Ein weiterer Pluspunkt ist, dass die Entwickler von Kotlin versucht haben aus den Fehlern von Java zu lernen und diese zu beseitigen. Ein großer Aspekt dabei ist die Sicherheit vor Nullwerten, welche dafür sorgt das keine \code{NullPointerException} mehr geworfen werden können. Diesen Fehler bezeichnet selbst Tony Hoare, der Entwickler der Null-Referenz, in seiner Präsentation \cite{billionDollorMistake} als \enquote{billion-dollar mistake}. Sollte diese Funktionalität dennoch notwendig sein, so kann der Datentyp einer Variable so definiert werden, dass dieser auch den Wert \code{null} annehmen kann \cite{kotlinNullSafety}. Eine ganze Reihe weiterer behobener Fehler bzw. Verbesserungen können auf der offiziellen Referenzseite \cite{kotlinComparisionJava} nachgelesen werden.\\
\\
Da Java, wie schon im \hyperref[sec:comparisonResults]{Kapitel~\ref{sec:comparisonResults}} erwähnt, seit 1995 existiert und sich sehr großer Beliebtheit erfreut, existieren logischerweise derzeit eine ganze Menge an Bibliotheken. Diese bei einem Umstieg zu Kotlin neu- bzw. umzuschreiben zu müssen wäre sehr aufwendig. Deshalb stellt Kotlin für dieses Problem die Möglichkeit der Interoperabilität mit Java-Code bereit. Damit ist es möglich, alle in Java geschriebenen Bibliotheken \enquote{out-of-the-box} zu nutzen.\\
\\
Letztlich lässt sich nach der Meinung des Autors sagen, dass für eine Programmierung auf der \gls{JVM} Kotlin Java vorgezogen werden kann.

\section{Verwendung von Kotlin für clientseitige Programmierung als Ersatz für JavaScript}\label{sec:conclusionKotlinJS}
Bei der Kompilierung in JavaScript-Code konnte Kotlin mich nicht ganz überzeugen. Prinzipiell ist die Erweiterung der Objektorientierung und der statischen Typisierung durch Kotlin sehr nützlich. Es stellt sich jedoch die Frage, ob dies durch den ECMAScript 6\footnote{Dieser Standard wird durch so gut wie alle gängigen Browser fast zu 100\% unterstützt. (siehe \url{https://caniuse.com/\#search=es6} (besucht am 10.07.2018))} Standard noch notwendig ist. Durch diesen wird zumindest die Objektorientierung und eine Reihe weiterer Features, welche Kotlin gegenüber dem EcmaScript 5 Standard besitzt, ergänzt.\\
\\
Genau wie bei der Kompilierung in Java-Bytecode ist es auch bei der clientseitigen Programmierung wichtig, bereits existierende Bibliotheken verwenden zu können. Kotlin bietet dafür, wie schon im \hyperref[sec:comparisonResults]{Kapitel~\ref{sec:comparisonResults}} erwähnt, mittels \gls{JsInterop} die Möglichkeit Programm-\glspl{API} manuell nachzubauen oder mit dem Tool \enquote{ts2kt} zu generieren. Es kann dabei aber vorkommen, dass der resultierende Kotlin-Code nicht ganz korrekt erzeugt wird und dadurch manuelle Optimierungen notwendig sind, was wiederum Mehraufwand bedeutet.\\
\\
Ein weiteres Problem stellt die Verwendung von Java-Bibliotheken in Kotlin bei der Kompilierung nach JavaScript dar. In einem Forumsbeitrag \cite{kotlinJsUseJava} bestätigt Dmitry Jemerov, dass die Konvertierung von Java- nach Kotlin- und schlussendlich nach JavaScript-Code kein 100\%ig automatisierter Prozess ist. Das bedeutet, dass im Client-Code keine Java-Bibliotheken verwendet werden können. Auf dieses Problem bin ich bei der Umsetzung des Schach-Clients ebenfalls gelegentlich gestoßen, da viele Kotlin-Bibliotheken Funktionen aus der Java-\gls{API} verwenden und so nicht eingesetzt werden konnten.\\
\\
Schlussendlich lässt sich sagen, dass es derzeit noch ein paar Stolpersteine gibt, wofür aber bereits teilweise Lösungen bzw. Lösungsansätze existieren. Nach Meinung des Autors hat Kotlin Potenzial und es lohnt sich die Entwicklung im Auge zu behalten. Derzeit würde der Autor von einer Verwendung innerhalb kleiner Projekte abraten, da sich dafür der etwas größere Konfigurationsaufwand nicht lohnt. Das betrifft beispielsweise die Konfiguration des Build-Tools. Für größere Projekte hingegen, welche von der Objekt-Orientierung und Typisierung profitieren können, lohnt es sich Kotlin in Betracht zu ziehen.