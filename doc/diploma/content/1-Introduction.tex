% !TeX encoding=utf8
% !TeX spellcheck = de_DE
% !TeX root = ../Diploma.tex

\chapter{Einleitung}
Laut einer Onlinestudie der ARD/ZDF-Medienkommission von 2017 \cite{onlineStudy2017,resultsOnlineStudy2017} ist die Nutzung des Internets im letzten Jahr um 6\% in Deutschland gestiegen. Dieser große Bedeutungszuwachs allein in Deutschland zeigt auch die immer mehr zunehmende Nachfrage nach Webinhalten. Was wiederum auch ein Indiz für die große Beliebheit der Skriptsprache JavaScript sein könnte, denn diese ist laut der letzten jährlichen Onlineumfrage der Stack Exchange Inc. \cite{developerSurvey2017} auf Platz 1 der beliebtesten Programmiersprachen. Dennoch kann es mitunter schwierig werden, große Applikationen mit JavaScript zu entwickeln. An dieser Stelle kommt die noch sehr junge Programmiersprache Kotlin ins Spiel.\\
\\
Kotlin ist eine statisch typisierte objektorientierte Programmiersprache, welche in Java-Bytecode, in JavaScript und in Nativen Code kompiliert werden kann. Des Weiteren ist Kotlin neben Java seit Mai 2017 offizielle Programmiersprache für Android \cite{kotlinAndroidOfficial}. Dadurch ist Kotlin sehr vielseitig und lässt sich für viele Anwendungsfälle einsetzen. \\
\\
Wie bereits erwähnt, ist Kotlin noch eine sehr junge Programmiersprache. Deshalb soll mit dieser Arbeit analysiert werden, wie sie sich für die Implementierung von Server- bzw. Client-Anwendungen eignet.\\
\\
TODO:\\
- Kapitelverlauf erläutern\\
- Vorraussetzungen\\