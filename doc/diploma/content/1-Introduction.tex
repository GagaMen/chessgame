% !TeX encoding=utf8
% !TeX spellcheck = de_DE
% !TeX root = ../Diploma.tex

\chapter{Einleitung}
Laut einer Onlinestudie der ARD/ZDF-Medienkommission von 2017 \cite{onlineStudy2017,resultsOnlineStudy2017} ist die Nutzung des Internets im Jahr 2017 um 6\% deutschlandweit gestiegen. Dieser große Bedeutungszuwachs allein in Deutschland zeigt die immer mehr größere Nachfrage nach Webinhalten. Ausdruck dafür könnte die große Beliebheit der Skriptsprache JavaScript sein. Denn diese ist laut der letzten jährlichen Onlineumfrage der Stack Exchange Inc. \cite{developerSurvey2017} auf Platz 1 der beliebtesten Programmiersprachen. Dennoch kann es mitunter schwierig werden, große Applikationen mit JavaScript zu entwickeln. An dieser Stelle kommt die noch sehr junge Programmiersprache Kotlin ins Spiel.\\
\\
Kotlin ist eine statisch typisierte objektorientierte Programmiersprache, welche in Java-Bytecode, JavaScript und Nativen Code kompiliert werden kann. Des Weiteren ist Kotlin neben Java seit Mai 2017 offizielle Programmiersprache für Android \cite{kotlinAndroidOfficial}. Dadurch ist Kotlin sehr vielseitig und lässt sich für viele Anwendungsbereiche einsetzen.

\section{Ziel der Arbeit}
Grundsätzlich besteht die Programmierung einer Web-Applikation aus der Erstellung eines Servers sowie eines Clients. Für die serverseitige Programmierung stehen dafür eine ganze Reihe verschiedener Programmiersprachen zur Verfügung, aber für die clientseitige Programmierung ist meistens die Skriptsprache JavaScript das Mittel der Wahl. Nun ist es aber umständlich Web-Applikationen im mehreren Programmiersprachen entwickeln zu müssen und es wäre wesentlich angenehmer nur eine einzige Sprache verwenden zu können.\\
\\
Mit Kotlin existiert eine noch sehr junge Programmiersprache, die diese Anforderung erfüllen soll. Das Ziel dieser Arbeit ist es deshalb Kotlin in dieser Hinsicht zu untersuchen. Dabei soll am Beispiel des bekannten Strategiespiels Schach ein Server und ein Client entwickelt werden. Die daraus resultierende Implementierung soll etwaige auftretende Probleme für diesen Anwendungsbereich aufzeigen, um so ein Fazit zum möglichen Einsatz von Kotlin für Server-Client-Anwendungen ziehen zu können.

\section{Aufbau der Arbeit}
Diese Arbeit ist in acht weitere Kapitel unterteilt. Dabei werden zunächst im \hyperref[chap:basics]{zweiten Kapitel} grundlegende Themen behandelt, welche für das Verständnis der nachfolgenden Abschnitte von Bedeutung sind. Im \hyperref[sec:comparison]{Kapitel drei} folgt ein Vergleich zwischen der Programmiersprache Kotlin und dem \gls{GWT}, mit dessen Hilfe Java- in JavaScript-Code kompiliert werden kann. Es bietet somit eine mögliche Alternative zu Kotlin, um mit einer Programmiersprache eine Server-Client-Anwendung zu entwickeln.\\
\\
Die konzeptionelle Ausarbeitung des Beispiels Schach soll in den Kapiteln \hyperref[sec:conceptServer]{vier} für den Server und \hyperref[sec:conceptClient]{fünf} für den Client erfolgen. Dabei werden alle benötigten Anforderungen und Designentscheidungen für die Umsetzung definiert. Auf Grundlage dieser Entwürfe befassen sich die Kapitel \hyperref[chap:implementationServer]{sechs} und \hyperref[chap:implementationClient]{sieben} mit der konkreten Erläuterung der Implementierung, wobei im \hyperref[chap:implementationServer]{sechsten} der Server und im \hyperref[chap:implementationClient]{siebten} der Client beleuchtet wird.\\
\\
Anschließend dient das \hyperref[chap:conclusion]{achte Kapitel} zur Präsentation des aus der Arbeit resultierenden Fazits, in welchen Rückschlüsse zum Einsatz von Kotlin für die Entwicklung von Server-Client-Anwendungen gezogen werden.
Abschließend rundet das \hyperref[chap:outlook]{Kapitel neun} die Arbeit, mit einem Ausblick auf Ideen zur Verbesserungen der Implementierung und für weiterführende wissenschaftliche Arbeiten ab.