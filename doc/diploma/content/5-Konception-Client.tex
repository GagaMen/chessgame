% !TeX encoding=utf8
% !TeX spellcheck = de_DE
% !TeX root = ../Diploma.tex

\chapter{Konzept des Clients}\label{sec:conceptClient}
In diesem Kapitel ist die Konzeption des Schachclients, welche die Grundlage für die Umsetzung selbigem bildet. Dabei werden im ersten Abschnitt die benötigten Anforderungen beschrieben, die der Client erfüllen muss. Im zweiten Abschnitt erfolgt die Visualisierung und Beschreibung der einzelnen Ansichten, welche für eine bequeme Nutzerinteraktion benötigt werden.

\section{Anforderungen}\label{sec:anforderungenClient}
Grundlegend soll der Client als Visualisierung des Servers dienen. Dafür ist eine Verwaltung von Playern und Matches bereitzustellen, inklusive der Möglichkeit zum Anlegen neuer bzw. Bearbeiten schon angelegter Einträge. Um anschließend auch Schach spielen zu können, muss der Client dafür eine komfortable Möglichkeit in Form eines virtuellen Schachbrettes bieten.\\
\\
Als Grundanforderung dafür muss der Client natürlich mit dem Server kommunizieren können. Dafür muss dieser Requests senden und die empfangenen Response-Nachrichten verarbeiten können. Da der Server für manche Request spezielle Parameter benötigt, wie zum Beispiel einen String in der \gls{SAN}, müssen diese gegebenenfalls ermittelt werden können.\\
\\
Für ein bequemes Spielerlebnis soll der Client ein gestartetes Match automatisch aktualisieren, sobald sich die Daten auf dem Server verändert haben. Um dies zu realisieren, ist ein einfaches Polling-Verfahren zu implementieren.\\
\\ 
Die letzte Grundanforderung ist eine innovative und benutzerfreundliche Bedienung der Anwendung, so dass der Nutzer keinerlei Kenntnisse außer den Schachregeln besitzen muss.

\section{Mockups der Client-Ansichten}\label{sec:views}
In diesem Abschnitt wird auf Basis der im \hyperref[sec:anforderungenClient]{Kapitel~\ref{sec:anforderungenClient}} definierten Anforderungen ein Konzept entwickelt. Ziel ist dabei die Erstellung von Mockups der einzelnen Ansichten, um die Implementierung zu vereinfachen bzw. zu beschleunigen.

\subsection{Startansicht}\label{sec:startView}
Die Startansicht ist der Ausgangspunkt für die Nutzer, welche beim Aufruf der Root-\gls{URL} zurückgegeben wird. Mithilfe der Startansicht wird dem Nutzer die Möglichkeit bereitgestellt zur Player-Ansicht bzw. zur Match-Ansicht zu wechseln. Dafür steht ihm jeweils ein Button zum Auslösen dieses Wechsels zur Verfügung.\\
\\
In der \hyperref[fig:startView]{Abbildung~\ref{fig:startView}} werden das zuvor beschriebene visualisiert.
\begin{figure}[htb]
	\includegraphics[width=0.234\textwidth]{images/start-view.png}
	\caption{Mockup: Startansicht des Clients}
	\label{fig:startView}
\end{figure}

\subsection{Player-Ansicht}\label{sec:playerView}
Dieser Ansicht stellt dem Nutzer die Möglichkeit bereit, um alle angelegten Player zu verwalten. Dafür wird ihm eine Tabelle für die Übersicht und ein Formular zum Anlegen neuer oder Bearbeiten bereits angelegter Player bereitgestellt. Über eine Spalte innerhalb der Tabelle stehen Buttons zur Verfügung, über welche ein Player bearbeitet oder gelöscht werden kann.\\
\\
Anhand dieser Beschreibung wurde das Mockup aus der \hyperref[fig:playerView]{Abbildung~\ref{fig:playerView}} als Visualisierung entworfen.
\begin{figure}[htb]
	\includegraphics[width=0.84\textwidth]{images/player-view.png}
	\caption{Mockup: Player-Ansicht des Clients}
	\label{fig:playerView}
\end{figure}

\subsection{Match-Ansicht}\label{sec:matchView}
Mithilfe der Match-Ansicht wird dem Nutzer eine Match-Verwaltung zur Verfügung gestellt. Um dabei die Konsistenz zu wahren, ist auch diese genau so aufgebaut wie die Player-Ansicht, mit dem Unterschied, dass ein Match nicht gelöscht aber dafür gestartet werden kann. Daher ändern sich geringfügig die Buttons in der Tabelle.\\
\\
Die Beschreibung dieser Ansicht wird durch die \hyperref[fig:matchView]{Abbildung~\ref{fig:matchView}} grafisch dargestellt.
\begin{figure}[htb]
	\includegraphics[width=0.84\textwidth]{images/match-view.png}
	\caption{Mockup: Match-Ansicht des Clients}
	\label{fig:matchView}
\end{figure}

\subsection{Ansicht eines gestarteten Matches}\label{sec:gameView}
Durch diese Ansicht wird dem Nutzer ermöglicht ein gestartetes Match zu spielen. Dafür wird in erster Linie ein Schachbrett bereitgestellt, auf welchem die Figuren dargestellt werden. Mittels \enquote{Drag \& Drop} kann der Nutzer anschließend Spielfiguren bewegen. Für eine einfachere Bedienung und zur Unterstützung des Verständnisses der Schachregeln werden alle möglichen Züge einer Figur hervorgehoben, sobald über diese mit der Maus gefahren wird. Da Informationen über bereits geschmissene Figuren oder welche Züge bisher getätigt wurden sehr hilfreich sein können, werden diese neben dem Schachbrett dargestellt.\\
\\
Auf Grundlage dieser Beschreibung wurde das Mockup aus der \hyperref[fig:gameView]{Abbildung~\ref{fig:gameView}} entwickelt.\\
\begin{figure}[htb]
	\includegraphics[width=0.84\textwidth]{images/game-view.png}
	\caption{Mockup: Ansicht eines gestarteten Matches}
	\label{fig:gameView}
\end{figure}


