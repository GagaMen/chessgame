% !TeX encoding=utf8
% !TeX spellcheck = de_DE
% !TeX root = ../Diploma.tex

\chapter{Implementation des Servers}
\section{Verwendete Bibliotheken/Frameworks}
Die verwendeten Bibliotheken sollen die Umsetzung des Projektes vereinfachen und beschleunigen. Dabei werden diese in den nachfolgenden \hyperref[sec:bibspring, sec:bibormlite]{Unterabschnitten~\ref{sec:bibspring} bis \ref{sec:bibormlite}} in den Punkten Zweck, Einrichtung und Verwendung näher betrachtet. 

\subsection{Spring}\label{sec:bibspring}
Spring wird als ein leichtgewichtiges Framework für die Umsetzung von Java Applikationen beschrieben. Dabei bezieht sich das leichtgewichtig nicht auf die Größe oder Anzahl der enthaltenen Klassen. Es ist eher als geringer Aufwand an Änderungen am eigenen Programmcode zu verstehen, um die Vorteile des Frameworks nutzen zu können.\footnote{siehe \cite{proSpring5}}\\
\\
Um Spring in einem Projekt einzubinden, müssen folgende Zeilen, zusätzlich zu denen welche Kotlin benötigt, in der Build-Datei \enquote{build.gradle} eingefügt werden:
\begin{lstlisting}[style=lstStyleFramed, caption={Einbindung des Spring Framework mithilfe von Gradle}]
buildscript {
repositories {
mavenCentral()
}
dependencies {
classpath "org.jetbrains.kotlin:kotlin-allopen:1.2.30"
classpath "org.springframework.boot:spring-boot-gradle-plugin:1.5.4.RELEASE"
}
}
apply plugin: "kotlin-spring"
apply plugin: "org.springframework.boot"

repositories {
mavenCentral()
}

dependencies {
compile "org.springframework.boot:spring-boot-starter-web"
}
\end{lstlisting}
Zu beachten ist dabei das Gradle nur eine Lösung für die Einbindung ist. Da aber für die Umsetzung ebenfalls Gradle verwendet wird, werden alle anderen Lösungen an dieser Stelle vernachlässigt.\\
\\ 
Für die Erstellung eines \gls{REST} \gls{API} muss zunächst ein Controller für eine Ressource angelegt werden, dabei ist es sinnvoll für jede einen eigenen Controller zu definieren. Innerhalb werden anschließend alle Einstiegspunkte erzeugt. Als Beispiel soll ein klassisches \textit{Hello World} dienen siehe \hyperref[lst:springcontroller]{Listing~\ref{lst:springcontroller}}. Dafür wird eine Klasse \textit{GreetingController} mit einer Funktion \textit{getGreeting} definiert. Als Parameter bekommt diese Funktion einen Namen übergeben, welcher standardmäßig den String \textit{World} hält. An dieser Stelle kommt das Spring Framework ins Spiel. Dieses stellt eine Reihe von Annotation bereit, wobei \textit{@RestController} einen Controller für das \gls{API}, \textit{@GetMapping} ein Einstiegspunkt (Request Methode: GET) und \textit{@RequestParam} einen Parameter für diese Funktion definiert. 
\begin{lstlisting}[style=lstStyleFramed, caption={Beispiel: Spring Controller}, label=lst:springcontroller]
import org.springframework.web.bind.annotation.*

@RestController
class GreetingController {
@GetMapping("/greeting")
fun getGreeting(@RequestParam name: String = "World"): String {
return "Hello $name!"
}
}
\end{lstlisting}
Abschließend muss nur noch der Startpunkt für die Applikation definiert werden. Dafür wird wieder mithilfe einer Annotation ein Klasse erzeugt, welche aber keinerlei Informationen benötigt. Anschließend muss diese in der Main-Funktion gerufen werden siehe \hyperref[lst:springmain]{Listing~\ref{lst:springmain}}.
\begin{lstlisting}[style=lstStyleFramed, caption={Beispiel: Spring Application Class}, label=lst:springmain]
@SpringBootApplication
class Application

fun main(args: Array<String>) {
SpringApplication.run(Application::class.java, *args)
}
\end{lstlisting}
Für eine detaillierte Beschreibung dieses Beispiels stehen auf den Internetseiten \cite{springTutorialKotlin} und \cite{springTutorial} weitere Informationen bereit.

\subsection{SQLite}\label{sec:bibsqlite}
SQLite ist eine OpenSource Bibliothek, welche ein dateibasiertes relationales Datenbanksystem bereitstellt. Der größte Unterschied zu anderen relationalen SQL-Datenbank besteht darin, das SQLite keinen separaten Serverprozess besitzt und somit als eingebettete Datenbank-Engine betrachtet werden kann. Alle Tabellen, Indizes, Trigger und Sichten einer Datenbank werden dabei in einem plattformunabhängigen Format in einer einzigen Datei gespeichert. Das bedeutet, dass Datenbankdateien bequem zwischen 32-Bit und 64-Bit-Systemen oder Little-Endian- und Big-Endian-Architekturen getauscht werden können.\footnote{siehe \cite{sqliteAbout}}\\
\\
Für die Einbindung von SQLite in ein Projekt, mithilfe der Datenbankschnittstelle \gls{JDBC}, sind folgende Zeile in der Build-Datei von Gradle vonnöten:
\begin{lstlisting}[style=lstStyleFramed, caption={Einbindung der Bibliothek SQLite mithilfe von Gradle}]
repositories {
mavenCentral()
}

dependencies {
compile group: 'org.xerial', name: 'sqlite-jdbc', version: '3.21.0.1'
}
\end{lstlisting}
Ein einfaches Beispiel für die Verbindung zu einer SQLite Datenbank, die Erstellung von Tabellen und für das Absenden von SQL-Abfragen stellt \cite{sqliteJDBCTutorial} ein Tutorial bereit. Da für die Implementierung eine \gls{ORM} Bibliothek verwendet werden soll\footnote{siehe Kapitel \ref{sec:bibormlite}}, wird eine nähere Betrachtung für die Verwendung von SQLite mithilfe des \gls{JDBC} Treibers vernachlässigt.

\subsection{ORMLite}\label{sec:bibormlite}
ORMLite ist ein OpenSource \gls{ORM} Projekt von Gray Watson, welches für Java entwickelt wurde, aber in Kotlin durch die Möglichkeit der Interoperabilität mit Java ebenfalls verwendet werden kann. Die Bibliothek unterstützt dabei eine Reihe von Datenbank-Systemen, wie zum Beispiel MySQL, Postgres oder SQLite.\\
\\
Um ORMLite in ein Gradle Projekt einzubinden müssen die Zeilen aus \hyperref[lst:includeORMLite]{Abbildung~\ref{lst:includeORMLite}} in die Build-Datei eingetragen werden. Dabei muss neben der Core-Bibliothek die \gls{JDBC}-Bibliothek von ORMLite eingebunden werden, welches für die Verbindung zur Datenbank zuständig ist. Da aber \gls{JDBC} mit mehreren Datenbank-Systemen kommunizieren kann muss noch, wie in \hyperref[sec:bibsqlite]{Kapitel~\ref{sec:bibsqlite}} für SQLite erläutert, der Datenbank-Treiber für das verwendete Datenbank-System eingebunden werden.
\begin{lstlisting}[style=lstStyleFramed, caption={Einbindung der Bibliothek ORMLite mithilfe von Gradle}, label=lst:includeORMLite]
repositories {
mavenCentral()
}

dependencies {
compile "com.j256.ormlite:ormlite-core:5.1"
compile "com.j256.ormlite:ormlite-jdbc:5.1"
}
\end{lstlisting}
Für die Persitierung einzelner Klassen können durch ORMLite bereitgestellte Annotationen verwendet werden. (siehe \hyperref[lst:ormPersistExample]{Abbildung~\ref{lst:ormPersistExample}})
\todo[inline]{ggf. Syntax-Highlighting einrichten}
\begin{lstlisting}[style=lstStyleFramed, caption={[Beispiel: Persistierung einer Klasse mittels ORMLite\protect\footnote{Quelle: \cite{ormlite}}]Beispiel: Persistierung einer Klasse mittels ORMLite\protect\footnotemark},label=lst:ormPersistExample]
@DatabaseTable(tableName = "accounts")
public class Account {
@DatabaseField(id = true)
private String name;

@DatabaseField(canBeNull = false)
private String password;
...
Account() {
// all persisted classes must define a no-arg constructor with at least package visibility
}
...    
}
\end{lstlisting}
\footnotetext{Quelle: \cite{ormlite}}
Der Zugriff auf die Datenbank erfolgt mittels \gls{DAO} Klassen, welche für jede Tabelle erzeugt werden müssen. Mit diesen \glspl{DAO} können anschließend Datensätze erstellt, bearbeitet und gelöscht werden. Zu dem stellen die \glspl{DAO} eine Reihe von Funktionen bereit um Datensätze abzufragen, wie zum Beispiel die Abfrage nach alle Datensätzen in der Datenbank oder nach einem bestimmten Objekt anhand seiner ID. Wenn diese Standard Funktionen nicht ausreichen besteht des weiteren die Möglichkeit komplexe Abfragen zu generieren mithilfe von sogenannten Query-Buildern.
Zur Veranschaulich der Verwendung von ORMLite zeigt die \hyperref[lst:ormliteUsageExample]{Abbildung~\ref{lst:ormliteUsageExample}} ein Minmalbeispiel.
\begin{lstlisting}[style=lstStyleFramed, caption={[Beispiel: Verwendung von ORMLite\protect\footnote{verändert nach \cite{ormlite}}] Beispiel: Verwendung von ORMLite\protect\footnotemark}, label=lst:ormliteUsageExample]
String databaseUrl = "jdbc:sqlite:path/to/account.db";
ConnectionSource connectionSource = new JdbcConnectionSource(databaseUrl);

Dao<Account,String> accountDao = DaoManager.createDao(connectionSource, Account.class);

TableUtils.createTable(connectionSource, Account.class);

String name = "Jim Smith";
Account account = new Account(name, "_secret");
accountDao.create(account);

Account account2 = accountDao.queryForId(name);
System.out.println("Account: " + account2.getPassword());

connectionSource.close();
\end{lstlisting}
\footnotetext{verändert nach \cite{ormlite}}
