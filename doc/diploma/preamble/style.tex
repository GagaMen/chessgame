%% -- style section selections -->
\DefineCodeSection[true]{StyleColors}
\DefineCodeSection[true]{StyleMath}
\DefineCodeSection[true]{StyleDiagrams}
\DefineCodeSection[true]{StyleScience}
\DefineCodeSection[true]{StyleText}
\DefineCodeSection[true]{StyleFootnote}
\DefineCodeSection[true]{StyleQuotes}
\DefineCodeSection[true]{StyleCiteBib}
\DefineCodeSection[true]{StyleFigures}
\DefineCodeSection[true]{StyleCaptions}
\DefineCodeSection[true]{StyleTables}
\DefineCodeSection[true]{StyleIndexes}
\DefineCodeSection[true]{StyleVerbatim}
\DefineCodeSection[true]{StyleFancy}
\DefineCodeSection[true]{StyleParagraph}
\DefineCodeSection[true]{StyleLineSpacing}
\DefineCodeSection[true]{StylePageLayout}
\DefineCodeSection[true]{StyleTitlepage}
\DefineCodeSection[true]{StyleHeadFoot}
\DefineCodeSection[true]{StyleHeadings}
\DefineCodeSection[true]{StyleHeadingsFonts}
\DefineCodeSection[true]{StyleHeadingsLayout}
\DefineCodeSection[true]{StyleLayoutTOC}
\DefineCodeSection[true]{StylePdf}
\DefineCodeSection[true]{StyleFixProblems}
%% <--------------------------------
 
% ~~~~~~~~~~~~~~~~~~~~~~~~~~~~~~~~~~~~~~~~~~~~~~~~~~~~~~~~~~~~~~~~~~~~~~~~
% Colors
% ~~~~~~~~~~~~~~~~~~~~~~~~~~~~~~~~~~~~~~~~~~~~~~~~~~~~~~~~~~~~~~~~~~~~~~~~
\BeginCodeSection{StyleColors}
\IfMultDefined{definecolor,colorlet}{%

% color of headings
%\definecolor{sectioncolor}{RGB}{0, 51, 153} % blue
%\definecolor{sectioncolor}{RGB}{0, 25, 152} % darker blue
\definecolor{sectioncolor}{RGB}{0, 0, 0}     % black
%
% Farbe fuer grau hinterlegte Boxen (fuer Paket framed.sty)
\definecolor{frameshadecolor}{gray}{0.90}

\definecolor{pdfanchorcolor}{named}{black}
\definecolor{pdfmenucolor}{named}{red}
\definecolor{pdfruncolor}{named}{cyan}

\SetTemplateDefinition{Target}{Web}{%
  \IfDefined{definecolor}{
    \definecolor{pdfurlcolor}{rgb}{0,0,0.6}
    \definecolor{pdffilecolor}{rgb}{0.7,0,0}
    \definecolor{pdflinkcolor}{rgb}{0,0,0.6}
    \definecolor{pdfcitecolor}{rgb}{0,0,0.6}
  }
}%
\SetTemplateDefinition{Target}{Print}{%
  \IfDefined{definecolor}{
    \definecolor{pdfurlcolor}{rgb}{0,0,0}
    \definecolor{pdffilecolor}{rgb}{0,0,0}
    \definecolor{pdflinkcolor}{rgb}{0,0,0}
    \definecolor{pdfcitecolor}{rgb}{0,0,0}
  }
}%

% Execute color definition defined by Target->Web
\UseDefinition{Target}{Web}

% table colors 
\colorlet{tablebodycolor}{white!100}
\colorlet{tablerowcolor}{gray!10}
\colorlet{tablesubheadcolor}{gray!30}
\colorlet{tableheadcolor}{gray!25}

}{} % End: \IfMultDefined{definecolor}
\EndCodeSection{StyleColors}
% ~~~~~~~~~~~~~~~~~~~~~~~~~~~~~~~~~~~~~~~~~~~~~~~~~~~~~~~~~~~~~~~~~~~~~~~~
% Math Settings
% ~~~~~~~~~~~~~~~~~~~~~~~~~~~~~~~~~~~~~~~~~~~~~~~~~~~~~~~~~~~~~~~~~~~~~~~~
\BeginCodeSection{StyleMath}

%%% print vector in bold
%\let\oldvec\vec
%\def\vec#1{{\boldsymbol{#1}}} % bold vector
%\newcommand{\ve}{\vec} %

%%% exchange greek symbols
\let\ORGvarepsilon=\varepsilon
\let\varepsilon=\epsilon
\let\epsilon=\ORGvarepsilon
%
% \let\ORGvarrho=\varrho
% \let\varrho=\rho
% \let\rho=\ORGvarrho
%
% \let\ORGvartheta=\vartheta
% \let\vartheta=\theta
% \let\theta=\ORGvartheta
%
% \let\ORGvarphi=\varphi
% \let\varphi=\phi
% \let\phi=\ORGvarphi
\EndCodeSection{StyleMath}
% ~~~~~~~~~~~~~~~~~~~~~~~~~~~~~~~~~~~~~~~~~~~~~~~~~~~~~~~~~~~~~~~~~~~~~~~~
% Science Settings
% ~~~~~~~~~~~~~~~~~~~~~~~~~~~~~~~~~~~~~~~~~~~~~~~~~~~~~~~~~~~~~~~~~~~~~~~~
\BeginCodeSection{StyleScience}

% style setup of siunitx
\IfDefined{sisetup}{%

%  detect-family,
%  detect-weight,  

\sisetup{%
  mode = math, % text is printed using a math font
  detect-all,
  separate-uncertainty=true,
}

\IfDefined{iflanguage}{%
  \iflanguage{ngerman}{%
    \sisetup{%
      exponent-product = \cdot,
      number-unit-separator=\text{\,},
      output-decimal-marker={\text{,}},
    }
  }
}

\let\nicefrac\sfrac

% Emulate units package, sort of
\NewDocumentCommand\unit{om}{%
  \IfNoValueTF{#1}
    {\si{#2}}
    {\SI{#1}{#2}}%
}
\NewDocumentCommand\unitfrac{omm}{%
  \IfNoValueTF{#1}
    {\si{\sfrac{#2}{#3}}}
    {\SI{#1}{\sfrac{#2}{#3}}}%
}

} % end: \IfDefined

\EndCodeSection{StyleScience}
% ~~~~~~~~~~~~~~~~~~~~~~~~~~~~~~~~~~~~~~~~~~~~~~~~~~~~~~~~~~~~~~~~~~~~~~~~
% diagrams
% ~~~~~~~~~~~~~~~~~~~~~~~~~~~~~~~~~~~~~~~~~~~~~~~~~~~~~~~~~~~~~~~~~~~~~~~~
\BeginCodeSection{StyleDiagrams}

% setup of package pgfplots
\IfPackagesLoaded{tikz,pgfplots}{%

% tikz/pgf
\pgfplotsset{width=0.8\textwidth,compat=1.5.1}
%% See pgfplotstable documentation (4.12.1) for further options
% set decimal point to comma for german text
\IfDefined{iflanguage}{
  \iflanguage{ngerman}{%
    \pgfplotsset{%
      every tick label/.append style={/pgf/number format/use comma}
%      x tick label style={/pgf/number format/use comma},%
%      y tick label style={/pgf/number format/use comma},%
%      z tick label style={/pgf/number format/use comma}%      
    }%
  }{} % end of \iflanguage
  % for all languages
  \pgfplotsset{%
  	every tick label/.append style={/pgf/number format/set thousands separator={\,}},
  	every node near coord/.append style={/pgf/number format/set thousands separator={\,}}
  }% 
}{} % end of \IfDefined



\definecolor{colorseriesRGB1}{RGB}{0,     0, 192}
\definecolor{colorseriesRGB2}{RGB}{192,   0,   0}
\definecolor{colorseriesRGB3}{RGB}{0  , 128,   0}
\definecolor{colorseriesRGB4}{RGB}{192,   0, 192}

\pgfplotscreateplotcyclelist{colorseries-rgb}{
  {colorseriesRGB1},
  {colorseriesRGB2},
  {colorseriesRGB3},
  {colorseriesRGB4},
}


\definecolor{colorseriesOffice1}{RGB}{ 49,  93, 152}
\definecolor{colorseriesOffice2}{RGB}{154,  50,  47}
\definecolor{colorseriesOffice3}{RGB}{117, 150,  57}
\definecolor{colorseriesOffice4}{RGB}{ 92,  67, 125}
\definecolor{colorseriesOffice5}{RGB}{211, 112,  40}
\definecolor{colorseriesOffice6}{RGB}{ 45, 134, 161}

\pgfplotscreateplotcyclelist{colorseries-office}{%
  {colorseriesOffice1},%
  {colorseriesOffice2},%
  {colorseriesOffice3},%
  {colorseriesOffice4},%
  {colorseriesOffice5},%
  {colorseriesOffice6},%
}

% color cycle list for bar plots
\pgfplotsset{
  /pgfplots/bar cycle list/.style={/pgfplots/cycle list={%
    {colorseriesOffice1!20!black,fill=colorseriesOffice1!80!white,mark=none},%
    {colorseriesOffice2!20!black,fill=colorseriesOffice2!80!white,mark=none},%
    {colorseriesOffice3!20!black,fill=colorseriesOffice3!80!white,mark=none},%
    {colorseriesOffice4!20!black,fill=colorseriesOffice4!80!white,mark=none},%
    {colorseriesOffice5!20!black,fill=colorseriesOffice5!80!white,mark=none},%
    {colorseriesOffice6!20!black,fill=colorseriesOffice6!80!white,mark=none},%
    }
  },
}


}{} % end if pgfplots

\EndCodeSection{StyleDiagrams}
% ~~~~~~~~~~~~~~~~~~~~~~~~~~~~~~~~~~~~~~~~~~~~~~~~~~~~~~~~~~~~~~~~~~~~~~~~
% text related 
% ~~~~~~~~~~~~~~~~~~~~~~~~~~~~~~~~~~~~~~~~~~~~~~~~~~~~~~~~~~~~~~~~~~~~~~~~
\BeginCodeSection{StyleText}

%% style of URL
\IfDefined{urlstyle}{
  \urlstyle{tt} %sf
}

% font used in margins by package marginnote
\IfDefined{marginfont}{
  \IfDefined{color}{
    \renewcommand*{\marginfont}{\color{red}\sffamily}
  }
}

% Options of enumitem
\IfDefined{setlist}{%
  \setlist{itemsep=0pt}
}%

\EndCodeSection{StyleText}

% ~~~~~~~~~~~~~~~~~~~~~~~~~~~~~~~~~~~~~~~~~~~~~~~~~~~~~~~~~~~~~~~~~~~~~~~~
% Footnotes
% ~~~~~~~~~~~~~~~~~~~~~~~~~~~~~~~~~~~~~~~~~~~~~~~~~~~~~~~~~~~~~~~~~~~~~~~~
\BeginCodeSection{StyleFootnote}

% separation text to footnote
\addtolength{\skip\footins}{\baselineskip}

% printed text between multible footnotes
\renewcommand*{\multfootsep}{,\nobreakspace}

% removed because of warning - requires more documentation
%\KOMAoptions{%   
%   footnotes=multiple% nomultiple
%}

% standard superscript numbers in footnotes
%\deffootnote%
%   [1em]% width of marker
%   {1.5em}% indentation (general)
%   {1em}% indentation (par)
%   {\textsubscript{\thefootnotemark}}%

% remove superscript numbers in footnotes
\deffootnote
  {1.5em}% indentation (general)
  {1em}% indentation (par)
  {\makebox[1.5em][l]{\thefootnotemark}}

%% Change intendation of footnote
%\setlength\footnotemargin{10pt}

% Limit space of footnotes to 10 lines
\setlength{\dimen\footins}{10\baselineskip}

% prevent continuation of footnotes 
% at facing page
\interfootnotelinepenalty=10000 

\EndCodeSection{StyleFootnote}

% ~~~~~~~~~~~~~~~~~~~~~~~~~~~~~~~~~~~~~~~~~~~~~~~~~~~~~~~~~~~~~~~~~~~~~~~~
% Quotes
% ~~~~~~~~~~~~~~~~~~~~~~~~~~~~~~~~~~~~~~~~~~~~~~~~~~~~~~~~~~~~~~~~~~~~~~~~
\BeginCodeSection{StyleQuotes}
\IfPackageLoaded{csquotes}{

% All facilities which take a 'cite' argument will not insert
% it directly. They pass it to an auxiliary command called \mkcitation
% which  may be redefined to format the citation.
\renewcommand*{\mkcitation}[1]{{\,}#1}
\renewcommand*{\mkccitation}[1]{ #1}

\SetBlockThreshold{2} % Number of Lines at which a blockquote is separated
                      % from the text.

\newenvironment{myquote}%
  {\begin{quote}\small}%
  {\end{quote}}%
\SetBlockEnvironment{myquote}
%\SetCiteCommand{} % Changes citation command

} %end: \IfPackageLoaded{csquotes}
\EndCodeSection{StyleQuotes}
% ~~~~~~~~~~~~~~~~~~~~~~~~~~~~~~~~~~~~~~~~~~~~~~~~~~~~~~~~~~~~~~~~~~~~~~~~
% Citations / Style of Bibliography
% ~~~~~~~~~~~~~~~~~~~~~~~~~~~~~~~~~~~~~~~~~~~~~~~~~~~~~~~~~~~~~~~~~~~~~~~~
\BeginCodeSection{StyleCiteBib}

% biblatex bibliography options
% !TeX encoding=utf8
% !TeX spellcheck = en-US

\IfPackageLoaded{biblatex}{%
	\ExecuteBibliographyOptions{%
%--- Sorting --- --- ---
	sorting=nty, % Sort by name, title, year.
	% other options: 
	% nty        Sort by name, title, year.
	% nyt        Sort by name, year, title.
	% nyvt       Sort by name, year, volume, title.
	% anyt       Sort by alphabetic label, name, year, title.
	% anyvt      Sort by alphabetic label, name, year, volume, title.
	% ynt        Sort by year, name, title.
	% ydnt       Sort by year (descending), name, title.
	% none       Do not sort at all. All entries are processed in citation order.
	% debug      Sort by entry key. This is intended for debugging only.
	%
	sortcase=true,
	sortcites=true, % do/do not sort citations according to bib	
%--- Dates --- --- ---
	date=comp,  % (short, long, terse, comp, iso8601)
%	origdate=
%	eventdate=
%	urldate=
%	alldates=
	datezeros=true, %
	dateabbrev=true, %
%--- General Options --- --- ---
%	maxnames=1,
%	minnames=1,
	maxbibnames=15,%
	maxcitenames=1,%
	uniquename=true,% (biber only)
	maxalphanames=1,% (biber only)
%	autocite= % (plain, inline, footnote, superscript) 
	autopunct=true,
	language=auto,
	block=none, % (none, space, par, nbpar, ragged)
	notetype=foot+end, % (foot+end, footonly, endonly)
	hyperref=true, % (true, false, auto)
	backref=true,
	backrefstyle=three, % (none, three, two, two+, three+, all+)
	backrefsetstyle=setonly, %
	indexing=false, % 
	% options:
	% true       Enable indexing globally.
	% false      Disable indexing globally.
	% cite       Enable indexing in citations only.
	% bib        Enable indexing in the bibliography only.
	refsection=none, % (part, chapter, section, subsection)
	refsegment=none, % (none, part, chapter, section, subsection)
	abbreviate=true, % (true, false)
	defernumbers=true, % 
	punctfont=false, % 
	arxiv=abs, % (ps, pdf, format)	
%--- Style Options --- --- ---	
% The following options are provided by the standard styles
	isbn=false,%
	url=true,%
	doi=false,%
	eprint=false,%	
	}%	
}% \IfPackageLoaded{biblatex}
% modifications for an alpha style
% !TeX encoding=utf8
% !TeX spellcheck = en-US

\IfPackageLoaded{biblatex}{%
% the number is not used in the bibliography, nor
% the citations, but for the list of publications
% we want numbers to be available.
\ExecuteBibliographyOptions{labelnumber}

% change alpha label to be without +	
\renewcommand*{\labelalphaothers}{}

% change 'In: <magazine>" to "<magazine>"
\renewcommand*{\intitlepunct}{}
\DefineBibliographyStrings{german}{in={}}
\DefineBibliographyStrings{english}{in={}}

% make names capitalized \textsc{}
\renewcommand{\mkbibnamefirst}{\textsc}
\renewcommand{\mkbibnamelast}{\textsc}

% make volume and number look like 
% 'Bd. 33(14): '
\renewbibmacro*{volume+number+eid}{%
\setunit{\addcomma\space}%
\bibstring{volume}% 
\setunit{\addspace}%
\printfield{volume}%
\iffieldundef{number}{}{% 
 \printtext[parens]{%
   \printfield{number}%
 }%
}%
\setunit{\addcomma\space}%
\printfield{eid}
%\setunit{\addcolon\space}%
}	

% <authors>: <title>
\renewcommand*{\labelnamepunct}{\addcolon\space}
% make ': ' before pages
\renewcommand*{\bibpagespunct}{\addcolon\space}
% names delimiter ';' instead of ','
%\renewcommand*{\multinamedelim}{\addsemicolon\space}

% move date before issue
\renewbibmacro*{journal+issuetitle}{%
\usebibmacro{journal}%
\setunit*{\addspace}%
\iffieldundef{series}
 {}
 {\newunit
  \printfield{series}%
  \setunit{\addspace}}%
%
\usebibmacro{issue+date}%
\setunit{\addcolon\space}%
\usebibmacro{issue}%
\setunit{\addspace}%
\usebibmacro{volume+number+eid}%
\newunit}

% print all names, even if maxnames = 1
\DeclareCiteCommand{\citeauthors}
{
\defcounter{maxnames}{1000}
\boolfalse{citetracker}%
\boolfalse{pagetracker}%
\usebibmacro{prenote}}
{\ifciteindex
  {\indexnames{labelname}}
  {}%
\printnames{labelname}}
{\multicitedelim}
{\usebibmacro{postnote}}

%% create a new style for an enumerated publication list
%% this code is taken from http://tex.stackexchange.com/questions/187181/independent-publication-list-with-numbered-list-using-biblatex-and-refsection

%% Emphasize own name in References with boldface

% Doc: xpatch.pdf
\usepackage{xpatch}% 

% \bibboldnames: etoolbox-list of names to typeset bold in \printbibliography
\newcommand*{\bibboldnames}{}

\newbibmacro*{name:bold}[2]{%
  \def\do##1{\ifstrequal{#1, #2}{##1}{\bfseries\listbreak}{}}%
  \dolistloop{\bibboldnames}}

%% # can not be used in patch command because the command is wrapped in another macro.
%% Therefore we mus play around with cat codes.
%%   see http://tex.stackexchange.com/questions/188188/loop-macro-fails-if-wrapped-in-conditional
%%   for a better explaination.
\begingroup\lccode`?=`#\lowercase{\endgroup
  \xpretobibmacro{name:last}{\begingroup\usebibmacro{name:bold}{?1}{?2}}{}{}
  \xpretobibmacro{name:first-last}{\begingroup\usebibmacro{name:bold}{?1}{?2}}{}{}
  \xpretobibmacro{name:last-first}{\begingroup\usebibmacro{name:bold}{?1}{?2}}{}{}
}%
\xpretobibmacro{name:delim}{\begingroup\normalfont}{}{}
\xapptobibmacro{name:last}{\endgroup}{}{}
\xapptobibmacro{name:first-last}{\endgroup}{}{}
\xapptobibmacro{name:last-first}{\endgroup}{}{}
\xapptobibmacro{name:delim}{\endgroup}{}{}

\DeclareNameAlias{default}{last-first/first-last}



% Define an new 'defbibenvironment'
% that includes numbers for use in extra refsections
\DeclareFieldFormat{labelnumberwidth}{#1\adddot}
\newlength{\periodwidth}
\settowidth{\periodwidth}{.}

\defbibenvironment{numbered+bold}
  {\list
     {\printtext[labelnumberwidth]{%
        \printfield{prefixnumber}%
        \printfield{labelnumber}%
        }%
     }%
  {
   \setlength{\labelwidth}{\labelnumberwidth}%
   \setlength{\leftmargin}{\labelwidth}%
   \setlength{\labelsep}{\biblabelsep}%
   \addtolength{\labelsep}{1em}
   \addtolength{\leftmargin}{\labelsep}%
   \setlength{\itemsep}{\bibitemsep}%
   \setlength{\parsep}{\bibparsep}}%
   \renewcommand*{\makelabel}[1]{\hss##1}%
  }
  {\endlist}
  {\item}%\hskip-\periodwidth
  
}% \IfPackageLoaded{biblatex}

% own styles
% necessary for long urls
\setcounter{biburllcpenalty}{7000}
\setcounter{biburlucpenalty}{8000}

\KOMAoptions{%
   % bibliography=oldstyle%
   bibliography=openstyle%
}%
\EndCodeSection{StyleCiteBib}
% ~~~~~~~~~~~~~~~~~~~~~~~~~~~~~~~~~~~~~~~~~~~~~~~~~~~~~~~~~~~~~~~~~~~~~~~~
% figures, placement and floats
% ~~~~~~~~~~~~~~~~~~~~~~~~~~~~~~~~~~~~~~~~~~~~~~~~~~~~~~~~~~~~~~~~~~~~~~~~
\BeginCodeSection{StyleFigures}
\IfPackageLoaded{float} {
% \floatplacement{figure}{H} % default placement
}

\IfPackageLoaded{wrapfig} {
%\setlength{\wrapoverhang}{\marginparwidth} 
%\addtolength{\wrapoverhang}{\marginparsep} 
\setlength{\intextsep}{0.5\baselineskip} % space above and below the image
% \intextsep ignored with draft ???
%\setlength{\columnsep}{1em} % separation to the text
}

% Make float placement easier
\renewcommand{\floatpagefraction}{.75} % previous: .5
\renewcommand{\textfraction}{.1}       % previous: .2
\renewcommand{\topfraction}{.8}        % previous: .7
\renewcommand{\bottomfraction}{.5}     % previous: .3
\setcounter{topnumber}{3}        % previous: 2
\setcounter{bottomnumber}{2}     % previous: 1
\setcounter{totalnumber}{5}      % previous: 3

\EndCodeSection{StyleFigures}
% ~~~~~~~~~~~~~~~~~~~~~~~~~~~~~~~~~~~~~~~~~~~~~~~~~~~~~~~~~~~~~~~~~~~~~~~~
% Captions
% ~~~~~~~~~~~~~~~~~~~~~~~~~~~~~~~~~~~~~~~~~~~~~~~~~~~~~~~~~~~~~~~~~~~~~~~~
\BeginCodeSection{StyleCaptions}

\IfPackageLoaded{amsmath}{
% Numbering of figures and table in each chapter
% \numberwithin{figure}{chapter}
% \numberwithin{table}{chapter}
}

% Style of captions and subcaptions (and subfig)
\IfPackageLoaded{caption}{%
% Style of captions
\DeclareCaptionStyle{captionStyleTemplateDefault}
[ % single line captions
  justification = centering
]
{ % multiline captions
% -- Formatting
  format    = plain,  % plain, hang
  indention  = 0em,   % indention of text 
  labelformat = default,% default, empty, simple, brace, parens
  labelsep   = colon,  % none, colon, period, space, quad, newline, endash
  textformat  = simple, % simple, period
% -- Justification
  justification = justified, %RaggedRight, justified, centering
  singlelinecheck = true, % false (true=ignore justification setting in 
%single line)
% -- Fonts
  labelfont  = {small,bf},
  textfont   = {small,rm},
% valid values:
% scriptsize, footnotesize, small, normalsize, large, Large
% normalfont, ip, it, sl, sc, md, bf, rm, sf, tt
% singlespacing, onehalfspacing, doublespacing
% normalcolor, color=<...>
%
% -- Margins and further paragraph options
  margin = 10pt, %.1\textwidth,
  % width=.8\linewidth,
% -- Skips
  skip    = 10pt, % vertical space between the caption and the figure
  position = auto, % top, auto, bottom
% -- Lists
  % list=no, % suppress any entry to list of figure 
  listformat = subsimple, % empty, simple, parens, subsimple, subparens
% -- Names & Numbering
  % figurename = Abb. %
  % tablename  = Tab. %
  % listfigurename=
  % listtablename=
  % figurewithin=chapter
  % tablewithin=chapter
%-- hyperref related options
  hypcap=true, % (true, false) 
  % true=all hyperlink anchors are placed at the 
  % beginning of the (floating) environment
  %
  hypcapspace=0.5\baselineskip
}

% apply caption style
\captionsetup{
  style = captionStyleTemplateDefault % base
}

% Predefinded skip setup for different floats
\captionsetup[table]{position=top}
\captionsetup[figure]{position=bottom}

\newcommand\FigureAbbrevition{Fig.}
\IfDefined{iflanguage}{%
  \iflanguage{ngerman}{%
    \renewcommand\FigureAbbrevition{Abb.}
  }{}
}

\DeclareCaptionStyle{captionStyleTemplateShortDefault}{%
  style=captionStyleTemplateDefault,
  name=\FigureAbbrevition,
  indention=0pt,
  justification=RaggedRight
}

% Short Names 
\IfDefined{wrapfigure}{%
  \captionsetup[wrapfigure]{style=captionStyleTemplateShortDefault}}
\IfDefined{wrapfloat}{%
  \captionsetup[wrapfloat]{style=captionStyleTemplateShortDefault}}
\IfDefined{floatingfigure}{%
  \captionsetup[floatingfigure]{style=captionStyleTemplateShortDefault}}
\IfDefined{margincap}{%
  \IfDefined{preto}{\preto\margincap{
  \captionsetup{style=captionStyleTemplateShortDefault}}}}
  % see http://tex.stackexchange.com/questions/37721/captionsetup-for-margin-caption
  % for an explanation of the extra code.
  %
} % end \IfPackageLoaded{caption}

% options for subcaptions
\IfPackageLoaded{subcaption}{
  \captionsetup[sub]{ %
    style = captionStyleTemplateDefault, % base
    labelfont  = {footnotesize,bf},
    textfont   = {footnotesize,rm},
    justification = RaggedRight, %RaggedRight, justified, centering
    skip=6pt,
    margin=5pt,
    labelformat = simple,% default, empty, simple, brace, parens
    labelsep    = space,
    list=false,
    hypcap=false
  }
  % make subcaptions be referenced as 5.3(b)
  \renewcommand\thesubfigure{(\alph{subfigure})} 
}

% style options for subfig
\IfPackageLoaded{caption}{%
 \IfPackageLoaded{subfig}{%
  \captionsetup[subfloat]{%
   style = captionStyleTemplateDefault, % base
   skip=6pt,
   margin=5pt,
   labelformat = parens,% default, empty, simple, brace
   labelsep    = space,
   list=false,
   hypcap=false
  }
 } % end \IfPackageLoaded{subfig}
} % end \IfPackageLoaded{caption}


% Style of figure placement with floatrow
\IfPackageLoaded{floatrow}{%

\floatsetup[table]{style=plaintop}

\DeclareFloatStyle{TemplateFloatStyleBoxed}%
   {style=Boxed,frameset={\fboxrule1pt\fboxsep12pt}}
   
\DeclareFloatVCode{grayruleabove}%
   {{\color{gray}\par\rule\hsize{2.8pt}\vskip4pt\par}}
   
\DeclareFloatVCode{grayrulebelow}%
   {{\color{gray}\par\vskip4pt\rule\hsize{2.8pt}}}
   
\DeclareColorBox{TemplateFloatColorBoxStyle}%
   {\fcolorbox{gray}{white}}
   
\DeclareObjectSet{centering}{\centering}

\DeclareMarginSet{center}%
   {\setfloatmargins{\hfil}{\hfil}}
   
\DeclareMarginSet{hangleft}%
   {\setfloatmargins{\hskip-\marginparwidth\hskip-\marginparsep}{\hfil}}
   
\DeclareFloatSeparators{marginparsep}%
   {\hskip\marginparsep}

\floatsetup{%
   %% style
   style={%
      plain % Standard LaTeX
      % plaintop % puts captions above float object's contents
      % Plaintop % Capitalized form of plaintop
      % ruled
      % Ruled
      % boxed
      % Boxed
      % BOXED
      % shadowbox
      % Shadowbox
      % SHADOWBOX
      % Doublebox
      % DOUBLEBOX
      % wshadowbox
      % Wshadowbox
      % WSHADOWBOX
   },%
   %%% --- Font --
   % uses caption-package formats
   % font=
   % footfont=
   %%% --- Position of Caption ---
%   capposition=top, % caption above object
%   %% caption above object and also aligned by top line in float row.
%   capposition=TOP, 
%   capposition=bottom, % caption below object
%   capposition=beside, % caption beside object.
%   %
%   %%% --- Position of Beside Caption ---
%   %% caption is printed to the left side of object
%   capbesideposition=left, 
%   %% caption is printed to the right side of object;
%   capbesideposition=right, 
%   % caption is printed in binding side of page if
%   % twoside option switched on in document class and key
%   % facing=yes is used; in oneside option of document
%   % (or key facing=no is used), caption is printed at the left side;
%   capbesideposition=inside,
%   capbesideposition=outside,
%   % least popular option: caption printed in outer side of page
%   % if twoside option switched on in document class and key
%   % facing=yes is used; in oneside option of document
%   % (or key facing=no is used), caption is printed at the right side.
%   capbesideposition=top, % caption aligned to the top of object;
%   capbesideposition=bottom, % caption aligned to the bottom of object;
%   capbesideposition=center, % caption aligned to the center of object.
%   %
%   capbesidewidth=4cm, % Defines width of beside caption.
%   floatwidth=7cm, % Defines width of objects
%   capbesideframe=no, % Align Caption at frame, not text
   %
   footposition=default, % if caption above float object foot material is placed
                         % below float object, otherwise below caption;
%   footposition=caption, % always placed below caption;
%   footposition=bottom,  % always placed at the bottom of float box.
   %
   %%% --- Vertical Alignment of Float Elements ---
   %% - heightadjust ----
   heightadjust={%
      %all, % adjust both caption and object heights
                     % (e.g. for styles ruled, Ruled and BOXED);
      % caption, % adjust caption heights (e.g. for Plaintop style);
      object, % adjust object heights (e.g. for Boxed style);
      % none, % nothing to be adjusted (the plain style);
      % nocaption, % no adjusting for captions;
      % noobject, % no adjusting for objects;
   },%
   %
   %% - valign ---
   % valign=t, % aligns objects by top line;
   % valign=c, % aligns objects by center line
   valign=b, % aligns objects by bottom line;
   % valign=s, % stretches objects by full height (if it is possible).
   %%% --- Facing Layout ---
   facing=yes, % different layout for even and odd pages in if twoside is on
   %%% --- Object Settings ---
   %% - objectset: Defines justification of float object (float contents).
   % objectset=justified,    %
   objectset=centering,    %
   % objectset=raggedright,  %
   % objectset=RaggedRight,  %
   %%% --- Defining Float Margins ---
   %% - margins: ????
   margins=centering,   %
   % margins=raggedright, %
   % margins=raggedleft,  %
   %%% --- Defining Float Separators ---
   % horizontal skip = \columnsep (default for both keys);
    floatrowsep=columnsep, 
   % floatrowsep=quad,  % horizontal skip = 1 em;
   % floatrowsep=qquad, % horizontal skip = 2 em;
   % floatrowsep=hfil,  % like \hfil
   % floatrowsep=hfill, % like \hfill
   % floatrowsep=none,  % empty separator
   %
   % horizontal skip = \columnsep (default for both keys);
   capbesidesep=columnsep, 
   % capbesidesep=quad,  % horizontal skip = 1 em;
   % capbesidesep=qquad, % horizontal skip = 2 em;
   % capbesidesep=hfil,  % like \hfil
   % capbesidesep=hfill, % like \hfill
   % capbesidesep=none,  % empty separator
   %%% --- Defining Float Rules/Skips ---
   %% - precode:     above float box
   precode={
      none %
      % thickrule %
      % rule %
      % lowrule %
      % captionskip
   },%
   %% - rowprecode:  above alone float box
   rowprecode={
      none %
      % thickrule %
      % rule %
      % lowrule %
      % captionskip
   },%
   %% - midcode:     between caption above/below and float object.
   midcode={%
      %none %
      % thickrule %
      % rule %
      % lowrule %
      captionskip
   },%
   %% - postcode:    below float box
   postcode={%
      none %
      % thickrule %
      % rule %
      % lowrule %
      % captionskip
   },%
   %% - rowpostcode: below alone float box
   rowpostcode={%
      none %
      % thickrule %
      % rule %
      % lowrule %
      % captionskip
   },%
   %%% --- Defining Float Frames ---
%   framestyle={%
%      % fbox %
%      colorbox %
%      % doublebox %
%      % shadowbox %
%      % wshadowbox %
%   },
   %% - frameset: The parameters for chosen frame
   % frameset={\fboxrule1pt\fboxsep12pt},
%   framearound={%
%      object % float object contents
%      % all % full float box
%   },
   framefit=yes, % fit frame to whatever is set
   %%% --- Settings for Colored Frames ---
   % Predefinded ColorBox (\DeclareColorBox)
%   colorframeset=TemplateFloatColorBoxStyle,
   %%% --- Defining Float Skips ---
   captionskip=5pt,
   footskip=\skip\footins,
   %%% --- Defining Float Footnote Rule's Style ---
   % Defines type of footnote rule for footnotes inside floating environment.
   footnoterule={
      normal   % standard LaTeX definition
      % limited  % standard LaTeX definition, max width of footnote \frulemax
      % fullsize % rule to full current text width.
      % none     % Absent rule.
   },
   %%% --- Managing Floats with [H] Placement Option ---
   % doublefloataswide=true, % ???
   % floatHaslist=false, % only true for backward compatibility
}


\floatsetup[FloatStyleCaptionMargin]{
  margins=hangleft,
  floatwidth=\textwidth,
  capposition=beside,
  capbesideposition=left,
  capbesideframe=no,
  capbesidewidth=\marginparwidth,
  capbesidesep=marginparsep,
  framestyle=framefit=yes,
}

%%% Replacement of <float> Package
%\DeclareNewFloatType{%
%   placement={%
%      tbh % any of t,b,h,p
%   },%
%   name={
%      % Defines the name of environment in the caption label.
%   },%
%   fileext={
%       % Defines extension of the file in which gathered list of floats.
%   }
%   within={% Reset caption within...
%      % nothing = do not reset ever
%      section % also section/chapter/part
%   },%
%   relatedcapstyle=yes % yes/no, related to \captionsetup
%}%

}% end if 

\EndCodeSection{StyleCaptions}
% ~~~~~~~~~~~~~~~~~~~~~~~~~~~~~~~~~~~~~~~~~~~~~~~~~~~~~~~~~~~~~~~~~~~~~~~~
% table packages
% ~~~~~~~~~~~~~~~~~~~~~~~~~~~~~~~~~~~~~~~~~~~~~~~~~~~~~~~~~~~~~~~~~~~~~~~~
\BeginCodeSection{StyleTables}

% for Package tabu
\IfDefined{tabulinesep}{%
  \tabulinesep=5pt
}

% Define new column types only if they are not yet defined
\IfDefined{RaggedLeft}{
  %% centered (Z):
  \IfColumntypeDefined{Z}{}
    {\newcolumntype{Z}{>{\Centering\arraybackslash\hspace{0pt}}X}}
  %% right (X):
  \IfColumntypeDefined{Y}{}
    {\newcolumntype{Y}{>{\RaggedLeft\arraybackslash\hspace{0pt}}X}}
  %% left (X):
  \IfColumntypeDefined{W}{}
    {\newcolumntype{W}{>{\RaggedRight\arraybackslash\hspace{0pt}}X}}
  %% left (p):
  \IfColumntypeDefined{L}{}
    {\newcolumntype{L}[1]{>{\RaggedRight\arraybackslash\hspace{0pt}}p{#1}}}
  %% right (p):
  \IfColumntypeDefined{R}{}
    {\newcolumntype{R}[1]{>{\RaggedLeft\arraybackslash\hspace{0pt}}p{#1}}}
  %% centered (p):
  \IfColumntypeDefined{C}{}
    {\newcolumntype{C}[1]{>{\Centering\arraybackslash\hspace{0pt}}p{#1}}}
}

\EndCodeSection{StyleTables}
% ~~~~~~~~~~~~~~~~~~~~~~~~~~~~~~~~~~~~~~~~~~~~~~~~~~~~~~~~~~~~~~~~~~~~~~~~
% Index and other lists
% ~~~~~~~~~~~~~~~~~~~~~~~~~~~~~~~~~~~~~~~~~~~~~~~~~~~~~~~~~~~~~~~~~~~~~~~~
\BeginCodeSection{StyleIndexes}

\IfPackageLoaded{imakeidx}{%

\indexsetup{%
  ,level=\chapter*%
  ,toclevel=chapter % indicate the level at which the indices appear in TOC
  ,noclearpage=false%
  ,firstpagestyle=plain%
  ,headers={\indexname}{\indexname}%
  ,othercode={\label{sec:Index}}% will be executed at the beginning of index entries typesetting
}%

}% end if \IfPackageLoaded
\IfPackageLoaded{glossaries}{%

% disable hyperref links for glossaries
\glsdisablehyper

% disable point at the end of each description
\renewcommand*{\glspostdescription}{}

\newglossarystyle{longFancy}{%
  \setglossarystyle{long}%
  \renewenvironment{theglossary}%
     {%
        \vspace*{-1\baselineskip}
        \renewcommand{\arraystretch}{1.6}%
        \normalfont\normalsize%
        \centering%           
        \rowcolors{1}{tablerowcolor}{tablebodycolor}
        \begin{longtable}{l>{\RaggedRight}p{\glsdescwidth}}%
     }%
     {\end{longtable}}%
  \renewcommand*{\glsgroupskip}{}%
  \renewcommand*{\glossaryheader}{%
    \hline\endhead%
    \hline\endfoot%
  }%  
}

\setlength{\glsdescwidth}{0.75\textwidth}

\newglossarystyle{longFancyHeader}{%
  \setglossarystyle{longFancy}%
  \renewcommand*{\glossaryheader}{%
    \hline\rowcolor{tableheadcolor}
      \bfseries \entryname & 
      \bfseries \descriptionname \tabularnewline
    \hline\endhead%
    \hline\endfoot%
    }%
}

\setglossarystyle{longFancyHeader}

\IfPackageLoaded{tabu}{%
  \newglossarystyle{longtabuFancy}{%
    \setglossarystyle{long}%
    \renewenvironment{theglossary}%
       {%
          \vspace*{-1\baselineskip}       
          \renewcommand{\arraystretch}{1.6}%
          \normalfont\normalsize%
          \centering%           
          \rowcolors{1}{tablerowcolor}{tablebodycolor}
            \begin{longtabu} to 0.95\textwidth{lX[L]}
       }%
       {\end{longtabu}}%
    \renewcommand*{\glsgroupskip}{}%
    \renewcommand*{\glossaryheader}{%
      \hline\endhead%
      \hline\endfoot%
    }%  
  } % end of newglossarystyle
  
  \newglossarystyle{longtabuFancyHeader}{%
    \setglossarystyle{longtabuFancy}%
    \renewcommand*{\glossaryheader}{%
      \hline\rowcolor{tableheadcolor}
        \bfseries \entryname & 
        \bfseries \descriptionname \tabularnewline
      \hline\endhead%
      \hline\endfoot%
      }%
  }
  \setglossarystyle{longtabuFancyHeader}
} % end of IfPackage


\IfPackageLoaded{translator}{% 
   \deftranslation[to=German]{Acronyms}{Abkürzungsverzeichnis}%
   \deftranslation[to=German]{List of Symbols}{Symbolverzeichnis}%
   \deftranslation[to=German]{Glossary}{Glossar}%
}% 

}% end if 

\EndCodeSection{StyleIndexes}
% ~~~~~~~~~~~~~~~~~~~~~~~~~~~~~~~~~~~~~~~~~~~~~~~~~~~~~~~~~~~~~~~~~~~~~~~~
% verbatim packages
% ~~~~~~~~~~~~~~~~~~~~~~~~~~~~~~~~~~~~~~~~~~~~~~~~~~~~~~~~~~~~~~~~~~~~~~~~
\BeginCodeSection{StyleVerbatim}

\IfDefined{colorlet}{
   \colorlet{colorlstNumber}{white!50!black!100}
}

\makeatletter
% see http://tex.stackexchange.com/questions/186651/spacefactor-problem-with-listings-in-a-special-case
% for the reason that \makeatletter must be placed outside the 
% \IfPackageLoaded conditional.
\IfPackageLoaded{listings}{%
%% provide command \addmoretexcs
% Code from Heiko Oberdiek, see
% http://tex.stackexchange.com/questions/84207/define-moretexcs-listings
% 
% Description:
% The following example defines macro \addmoretexcs. The optional argument
%  specifies the dialect (default is common). The language is loaded if it is
%  not yet available. Then the language definition, internally stored in
%  \lstlang@<language>$<dialect>, is extended by setting the additional
%  moretexcs list
% ------------>
\newcommand*{\addmoretexcs}[2][common]{%
  \lowercase{\@ifundefined{lstlang@tex$#1}}{%
    \lstloadlanguages{[#1]TeX}%
  }{}%
  \lowercase{\expandafter\g@addto@macro\csname lstlang@tex$#1\endcsname}{%
    \lstset{moretexcs={#2}}%
  }%
}
% <------------
}% End of \IfPackageLoaded
\makeatother

\IfPackageLoaded{listings}{%

\lstdefinestyle{lstStyleBase}{
%%% appearance
   ,basicstyle=\small\ttfamily % Standardschrift
%%%  Space and placement
   ,floatplacement=tbp    % is used as float place specifier
   ,aboveskip=\medskipamount % define the space above and 
   ,belowskip=\medskipamount % below displayed listings.
   ,lineskip=0pt          % specifies additional space between lines in listings.
   ,boxpos=c              % c,b,t
%%% The printed range
   ,showlines=false       % prints empty lines at the end of listings
%%% characters
   ,extendedchars=true   % allows or prohibits extended characters 
                         % in listings, that means (national)
                         % characters of codes 128-255. 
   ,upquote=true         % determines printing of quotes
   ,tabsize=2,           % chars of tab
   ,showtabs=false       % do not show tabs
   ,showspaces=false     % do not show spaces
   ,showstringspaces=false % do not show blank spaces in string
%%% Line numbers
   ,numbers=left         % left, right, none
   ,stepnumber=1         % seperation between numbers
   ,numberfirstline=false % number first line always
   ,numberstyle=\tiny\color{colorlstNumber}    % style of numbers
   ,numbersep=5pt        % distance to text
   ,numberblanklines=true %
%%% Captions
   ,numberbychapter=true %
   ,captionpos=b         % t,b
   ,abovecaptionskip=\smallskipamount % the vertical space respectively above 
   ,belowcaptionskip=\smallskipamount % or below each caption
%%% Margins and line shape
   ,linewidth=\linewidth % defines the base line width for listings.  
   ,xleftmargin=0pt      % extra margins
   ,xrightmargin=0pt     %
   ,resetmargins=false   % indention from list environments like enumerate 
                         % or itemize is reset, i.e. not used.
   ,breaklines=true      % line breaking of long lines.
   ,breakatwhitespace=false % allows line breaks only at white space.
   ,breakindent=0pt     % is the indention of the second, third, ...  
                         % line of broken lines.
   ,breakautoindent=true % apply intendation
   ,columns=flexible     %
   ,keepspaces=true      %
}

\lstset{style=lstStyleBase}

\lstdefinestyle{lstStyleFramed}{%
%%% Frames
   ,frame=single         % none, leftline, topline, bottomline, lines
                         % single, shadowbox
   ,framesep=3pt 
   ,rulesep=2pt          % control the space between frame and listing 
                         % and between double rules.
   ,framerule=0.4pt      % controls the width of the rules.
}

% do not activate!:
% frames in fancyvrb are printed out wrong!
% \IfPackageLoaded{fancyvrb}{\lstset{fancyvrb=true}}

% correct utf8 umlaute
\lstset{literate=%
  {Ö}{{\"O}}1
  {Ä}{{\"A}}1
  {Ü}{{\"U}}1
  {ß}{{\ss}}2
  {ü}{{\"u}}1
  {ä}{{\"a}}1
  {ö}{{\"o}}1
  {€}{{\geneuro{}}}1
}

\IfDefined{colorlet}{
  % style files make use of colors and require \colorlet
  \colorlet{lstcolorStringLatex}{green!40!black!100}
\colorlet{lstcolorCommentLatex}{green!50!black!100}
\definecolor{lstcolorKeywordLatex}{rgb}{0,0.47,0.80}

% define useless command for checking the
% existens of this style
\newcommand{\lstStyleLaTeX}{\relax}
% define style
\lstdefinestyle{lstStyleLaTeX}{%
   ,style=lstStyleBase
%%% colors
   ,stringstyle=\color{lstcolorStringLatex}%
   ,keywordstyle=\color{lstcolorKeywordLatex}%
   ,commentstyle=\color{lstcolorCommentLatex}%
   ,% backgroundcolor=\color{codebackcolor}%
%%% Frames
   ,frame=single%
   ,%frameround=tttt%
   ,%framesep = 10pt%
   ,%framerule = 0pt%
   ,rulecolor = \color{black}%
%%% language
   ,language = [LaTeX]TeX%
%%% commands
% moved to: listings-latex-texcs.tex
}

\ifcsdef{addmoretexcs}{%
% LaTeX programming
\addmoretexcs[LaTeX]{setlength,usepackage,newcommand,renewcommand,providecommand,RequirePackage,SelectInputMappings,ifpdftex,ifpdfoutput,AtBeginEnvironment,ProvidesPackage}
% other commands
\addmoretexcs[LaTeX]{maketitle,text,includegraphics,chapter,section,subsection,
subsubsection,paragraph,textmu,enquote,blockquote,ding,mathds,ifcsdef,Bra,Ket,Braket,subcaption,lettrine,mdfsetup,captionof,listoffigures,listoftables,tableofcontents,appendix,url}
% tables
\addmoretexcs[LaTeX]{newcolumntype,rowfont,taburowcolors,rowcolor,rowcolors,bottomrule,toprule,midrule}
% hyperref
\addmoretexcs[LaTeX]{hypersetup}
% glossaries
\addmoretexcs[LaTeX]{gls,printglossary,glsadd,newglossaryentry,newacronym}
% Koma
\addmoretexcs[LaTeX]{mainmatter,frontmatter,geometry,KOMAoptions,setkomafont,addtokomafont}
% SI, unit
\addmoretexcs[LaTeX]{si,SI,sisetup,unit,unitfrac,micro}
% biblatex package
\addmoretexcs[LaTeX]{newblock,ExecuteBibliographyOptions,addbibresource}
% math packages
\addmoretexcs[LaTeX]{operatorname,frac,sfrac,dfrac,DeclareMathOperator,mathcal,underset}
% demo package
\addmoretexcs[LaTeX]{democodefile,package,cs,command,env,DemoError,PrintDemo}  
% tablestyles
\addmoretexcs[LaTeX]{theadstart,tbody,tsubheadstart,tsubhead,tend}
% code section package
\addmoretexcs[LaTeX]{DefineCodeSection,SetCodeSection,BeginCodeSection,EndCodeSection}
% template tools package
\addmoretexcs[LaTeX]{IfDefined,IfUndefined,IfElseDefined,IfElseUndefined,IfMultDefined,IfNotDraft,IfNotDraftElse,IfDraft,IfPackageLoaded,IfElsePackageLoaded,IfPackageNotLoaded,IfPackagesLoaded,IfPackagesNotLoaded,ExecuteAfterPackage,ExecuteBeforePackage,IfTikzLibraryLoaded,IfColumntypeDefined,IfColumntypesDefined,IfColorDefined,IfColorsDefined,IfMathVersionDefined,SetTemplateDefinition,UseDefinition,IfFileExists,iflanguage}
% tablestyles
\addmoretexcs[LaTeX]{setuptablefontsize,tablefontsize,setuptablestyle,tablestyle,setuptablecolor,tablecolor,disablealternatecolors,tablealtcolored,tbegin,tbody,tend,thead,theadstart,tsubheadstart,tsubhead,theadrow,tsubheadrow,resettablestyle,theadend,tsubheadend,tableitemize,PreserveBackslash}
% todonotes
\addmoretexcs[LaTeX]{todo,missingfigure}
% listings
\addmoretexcs[LaTeX]{lstloadlanguages,lstdefinestyle,lstset}
% index
\addmoretexcs[LaTeX]{indexsetup}
% glossaries
\addmoretexcs[LaTeX]{newglossarystyle,glossarystyle,deftranslation,newglossary}
% tikz
\addmoretexcs[LaTeX]{usetikzlibrary}
% color
\addmoretexcs[LaTeX]{definecolor,colorlet}
% caption
\addmoretexcs[LaTeX]{captionsetup,DeclareCaptionStyle}
% floatrow
\addmoretexcs[LaTeX]{floatsetup}
% doc.sty
\addmoretexcs[LaTeX]{EnableCrossrefs,DisableCrossrefs,PageIndex,CodelineIndex,CodelineNumbered}
% refereces
\addmoretexcs[LaTeX]{cref,Cref,vref,eqnref,figref,tabref,secref,chapref}
%
}{} % end of \ifcsdef

\lstloadlanguages{[LaTeX]TeX}
  %\colorlet{colorlstStringCpp}{green!40!black!100}
\colorlet{colorlstCommentCpp}{green!50!black!100}
\colorlet{colorlstBackgroundCpp}{white!100}
\definecolor{colorlstStringCpp}{rgb}{0,0.47,0.80}

%% \colorlet{colorlstStringCpp}{green!100!black!100}
%% \colorlet{commencolor}{green!100!red!50!black!100}
%\definecolor{commencolor}{rgb}{0.0,0.5,0.0}
\definecolor{colorlstKeywordCpp}{rgb}{0.4,0.4,0.0}

% define useless command for checking the
% existens of this style
\newcommand{\lstStyleCpp}{\relax}
% define style
 \lstdefinestyle{lstStyleCpp}{%
   ,style=lstStyleBase
%%% Numbers
   ,,stepnumber=1%
%%% colors
   ,keywordstyle=\textbf\ttfamily\color{colorlstKeywordCpp}%
   ,identifierstyle=\ttfamily%
   ,commentstyle=\color{colorlstCommentCpp}%
   ,stringstyle=\ttfamily\color{colorlstStringCpp} %\color[rgb]{0,0.5,0},
   ,backgroundcolor=\color{colorlstBackgroundCpp}%
%%% Frames
   ,frame=single%
   ,%frameround=tttt
   ,%framesep = 10pt
   ,%framerule = 0pt
%%% language
   ,language = C++%
   ,otherkeywords={string},
%%% Comments
   ,morecomment=[l][\color{colorlstCommentCpp}]{//},%
   ,morecomment=[s][\color{colorlstCommentCpp}]{/*}{*/}%
}
 
\lstloadlanguages{
   ,C++
   ,[Visual]C++
   ,[ISO]C++
}
}

% load language used in document. 
% (LaTex and C++ already loaded)
\lstloadlanguages{
   %,[LaTeX]TeX
   %,C++
   %,[Visual]C++
   %,[ISO]C++   
   %,[Visual]Basic
   %,Pascal
   %,C
   %,XML
   %,HTML
}

}% End of \IfPackageLoaded

\EndCodeSection{StyleVerbatim}
% ~~~~~~~~~~~~~~~~~~~~~~~~~~~~~~~~~~~~~~~~~~~~~~~~~~~~~~~~~~~~~~~~~~~~~~~~
% fancy packages
% ~~~~~~~~~~~~~~~~~~~~~~~~~~~~~~~~~~~~~~~~~~~~~~~~~~~~~~~~~~~~~~~~~~~~~~~~
\BeginCodeSection{StyleFancy}
\IfPackageLoaded{lettrine}{
  \setcounter{DefaultLines}{2}
  \renewcommand{\DefaultLoversize}{0}
  \renewcommand{\DefaultLraise}{0}
  \renewcommand{\DefaultLhang}{0}
  \LettrineImagefalse
  \setlength{\DefaultFindent}{0pt}
  \setlength{\DefaultNindent}{0.5em}
  \setlength{\DefaultSlope}{0pt}
}

\IfPackageLoaded{framed}{
  \renewcommand\FrameCommand{\fcolorbox{black}{frameshadecolor}}
}
\EndCodeSection{StyleFancy}
% ~~~~~~~~~~~~~~~~~~~~~~~~~~~~~~~~~~~~~~~~~~~~~~~~~~~~~~~~~~~~~~~~~~~~~~~~
% layout:  Paragraph
% ~~~~~~~~~~~~~~~~~~~~~~~~~~~~~~~~~~~~~~~~~~~~~~~~~~~~~~~~~~~~~~~~~~~~~~~~
\BeginCodeSection{StyleParagraph}
%\nonfrenchspacing     % provides extra space after sentence endings 
                       % Must be switched of for german and english text!

%% Paragraph Separation =================================
\KOMAoptions{%
   % parskip=relative, % _not_ compatible with tikz! othwise recommanded
   parskip=absolute, % do not change indentation according to fontsize
   parskip=false    % indentation of 1em
   % parskip=true   % parksip of 1 line - with free space in last line of 1em
   % parskip=full-  % parksip of 1 line - no adjustment
   % parskip=full+  % parksip of 1 line - with free space in last line of 1/4
   % parskip=full*  % parksip of 1 line - with free space in last line of 1/3
   % parskip=half   % parksip of 1/2 line - with free space in last line of 1em
   % parskip=half-  % parksip of 1/2 line - no adjustment
   % parskip=half+  % parksip of 1/2 line - with free space in last line of 1/3
   % parskip=half*  % parksip of 1/2 line - with free space in last line of 1em
}%
\EndCodeSection{StyleParagraph}
% ~~~~~~~~~~~~~~~~~~~~~~~~~~~~~~~~~~~~~~~~~~~~~~~~~~~~~~~~~~~~~~~~~~~~~~~~
% layout:  line spacing
% ~~~~~~~~~~~~~~~~~~~~~~~~~~~~~~~~~~~~~~~~~~~~~~~~~~~~~~~~~~~~~~~~~~~~~~~~
%
\BeginCodeSection{StyleLineSpacing}
\IfPackageLoaded{setspace}{
  \onehalfspacing    % 1,5-times spacing
  %\doublespacing     % 2-times spacing
}
\EndCodeSection{StyleLineSpacing}
% ~~~~~~~~~~~~~~~~~~~~~~~~~~~~~~~~~~~~~~~~~~~~~~~~~~~~~~~~~~~~~~~~~~~~~~~~
% layout:  page layout
% ~~~~~~~~~~~~~~~~~~~~~~~~~~~~~~~~~~~~~~~~~~~~~~~~~~~~~~~~~~~~~~~~~~~~~~~~
%
\BeginCodeSection{StylePageLayout}

%\raggedbottom     % allow variable (ragged) site heights

% Layout with 'geometry'
\IfPackageLoaded{geometry}{%
  \geometry{%
%%% Paper Groesse
   a4paper, % Andere a0paper, a1paper, a2paper, a3paper, , a5paper, a6paper,
            % b0paper, b1paper, b2paper, b3paper, b4paper, b5paper, b6paper
            % letterpaper, executivepaper, legalpaper
   %screen,  % a special paper size with (W,H) = (225mm,180mm)
   %paperwidth=,
   %paperheight=,
   %papersize=, %{ width , height }
   %landscape,  % Querformat
   portrait,    % Hochformat
%%% Koerper Groesse
   %hscale=,      % ratio of width of total body to \paperwidth
                  % hscale=0.8 is equivalent to width=0.8\paperwidth. (0.7 by default)
   %vscale=,      % ratio of height of total body to \paperheight
                  % vscale=0.9 is equivalent to height=0.9\paperheight.
   %scale=,       % ratio of total body to the paper. scale={ h-scale , v-scale }
   %totalwidth=,    % width of total body % (Generally, width >= textwidth)
   %totalheight=,   % height of total body, excluding header and footer by default
   %total=,        % total={ width , height }
   % value similar to koma script with DIV=12
   textwidth=426.8pt,    % modifies \textwidth, the width of body
   textheight=595.8pt,   % modifies \textheight, the height of body
   %body=,        % { width , height } sets both \textwidth and \textheight of the body of page.
   %lines=45,       % enables users to specify \textheight by the number of lines.
   %includehead,  % includes the head of the page, \headheight and \headsep, into total body.
   %includefoot,  % includes the foot of the page, \footskip, into body.
   %includeheadfoot, % sets both includehead and includefoot to true
   %includemp,    % includes the margin notes, \marginparwidth and \marginparsep, into body
   %includeall,   % sets both includeheadfoot and includemp to true.
   %ignorehead,   % disregards the head of the page, headheight and headsep in determining vertical layout
   %ignorefoot,   % disregards the foot of page, footskip, in determining vertical layout
   %ignoreheadfoot, % sets both ignorehead and ignorefoot to true.
   %ignoremp,     % disregards the marginal notes in determining the horizontal margins
   %ignoreall,     % sets both ignoreheadfoot and ignoremp to true
   heightrounded, % This option rounds \textheight to n-times (n: an integer) of \baselineskip
   %hdivide=,     % { left margin , width , right margin }
                  % Note that you should not specify all of the three parameters
   %vdivide=,     % { top margin , height , bottom margin }
   %divide=,      % ={A,B,C} %  is interpreted as hdivide={A,B,C} and vdivide={A,B,C}.
%%% Margin
   %left=,        % left margin (for oneside) or inner margin (for twoside) of total body
                  % alias: lmargin, inner
   %right=,       % right or outer margin of total body
                  % alias: rmargin outer
   % set \oddsidemargin to 3.6pt 
   %  can not be set directly, must be calculated:
   %  inner = 1inch - bindingoffset + oddsidemargin
   inner=\dimexpr1in-10mm+3.6pt\relax,
   % set top (sets multiple values, for example \topmargin)
   %  such that it matches typearea with DIV 12 approx.
   top = 120pt, 
   %top=,         % top margin of the page.
                  % Alias : tmargin
   %bottom=,      % bottom margin of the page
                  % Alias : bmargin
   %hmargin=,     % left and right margin. hmargin={ left margin , right margin }
   %vmargin=,     % top and bottom margin. vmargin={ top margin , bottom margin }
   %margin=,      % margin={A,B} is equivalent to hmargin={A,B} and vmargin={A,B}
   %hmarginratio, % horizontal margin ratio of left (inner) to right (outer).
   %vmarginratio, % vertical margin ratio of top to bottom.
   %marginratio,  % marginratio={ horizontal ratio , vertical ratio }
   %hcentering,   % sets auto-centering horizontally and is equivalent to hmarginratio=1:1
   %vcentering,   % sets auto-centering vertically and is equivalent to vmarginratio=1:1
   %centering,    % sets auto-centering and is equivalent to marginratio=1:1
   twoside,       % switches on twoside mode with left and right margins swapped on verso pages.
   %asymmetric,   % implements a twosided layout in which margins are not swapped on alternate pages
                  % and in which the marginal notes stay always on the same side.
   bindingoffset=10mm,  % removes a specified space for binding
%%% Dimensionen
   headheight=28.5pt,  % Alias:  head
   %headsep=,     % separation between header and text
   %footskip=,    % distance separation between baseline of last line of text and baseline of footer
                  % Alias: foot
   %nohead,       % eliminates spaces for the head of the page
                  % equivalent to both \headheight=0pt and \headsep=0pt.
   %nofoot,       % eliminates spaces for the foot of the page
                  % equivalent to \footskip=0pt.
   %noheadfoot,   % equivalent to nohead and nofoot.
   %footnotesep=, % changes the dimension \skip\footins,.
                  % separation between the bottom of text body and the top of footnote text
   %marginparwidth=22pt, % width of the marginal notes
                  % Alias: marginpar
   %marginparsep=,% separation between body and marginal notes.
   %nomarginpar,  % shrinks spaces for marginal notes to 0pt
   %columnsep=,   % the separation between two columns in twocolumn mode.
   %hoffset=,
   %voffset=,
   %offset=,      % horizontal and vertical offset.
                  % offset={ hoffset , voffset }
   %twocolumn,    % twocolumn=false denotes onecolumn
   twoside,
   %reversemp,    % makes the marginal notes appear in the left (inner) margin
                  % Alias: reversemarginpar
}

} % Endif


%%% === Page Layout Options ===
\KOMAoptions{% 
   %
   headlines=2.1,%
   % headheight=2em,%
   cleardoublepage=empty %plain, headings
}%

% Layout with 'typearea'
%%% Doc: scrguide.pdf
\IfPackageLoaded{typearea}{% If typearea is loaded
   \IfPackageNotLoaded{geometry}{% and geometry is not loaded
     % Koma Script text area layout
     \KOMAoptions{%
        DIV=12,% (Size of Text Body, higher values = greater textbody)
        % DIV=calc % (also areaset/classic/current/default/last) 
        % -> after setting of spacing necessary!   
        BCOR=10mm% (binding correction)
     }%
 
     \KOMAoptions{% (most options are for package typearea)
       twoside=true, % two side layout (alternating margins, standard in books)
       % twoside=false, % single side layout 
       % twoside=semi,  % two side layout (non alternating margins!)
       %
       twocolumn=false, % (true)
       %
       headinclude=false,%
       footinclude=false,%
       mpinclude=false,%    
       headsepline=true,%
       footsepline=false,%
     }%
     % reloading of typearea, necessary if setting of spacing changed 
     \typearea[current]{last}      
%
% BCOR
%    current  % Recalculate type-area with the currently valid BCOR value.
%
% DIV
%    areaset  % Recalculate page layout.
%
%    calc     % Recalculate type-area including choice of appropriate DIV
%             % value.
%
%    classic  % Recalculate type-area using Middle Age book design canon
%             % (circle-based calculation).
%
%    current  % Recalculate type-area using current DIV value.
%
%    default  % Recalculate type-area using the standard value for the current
%             % page format and current font size. If no standard value
%             % exists, calc is used.
%
%    last     % Recalculate type-area using the same DIV argument as was used
%             % in the last call.
%
   } % \IfPackageNotLoaded{geometry}
} % \IfPackageLoaded{typearea}
\EndCodeSection{StylePageLayout}
% ~~~~~~~~~~~~~~~~~~~~~~~~~~~~~~~~~~~~~~~~~~~~~~~~~~~~~~~~~~~~~~~~~~~~~~~~
% Titlepage
% ~~~~~~~~~~~~~~~~~~~~~~~~~~~~~~~~~~~~~~~~~~~~~~~~~~~~~~~~~~~~~~~~~~~~~~~~
\BeginCodeSection{StyleTitlepage}
\KOMAoptions{%
   titlepage=true % % separate page for title
   %titlepage=false %
}%
\EndCodeSection{StyleTitlepage}
% ~~~~~~~~~~~~~~~~~~~~~~~~~~~~~~~~~~~~~~~~~~~~~~~~~~~~~~~~~~~~~~~~~~~~~~~~
% head and foot lines
% ~~~~~~~~~~~~~~~~~~~~~~~~~~~~~~~~~~~~~~~~~~~~~~~~~~~~~~~~~~~~~~~~~~~~~~~~
\BeginCodeSection{StyleHeadFoot}

\IfPackageLoaded{scrpage2}{%

\IfElseDefined{chapter}{%
   \pagestyle{scrheadings} % pages with header
}{
   \pagestyle{scrplain} % pages without header but page numbers
}
%\pagestyle{empty} % empty pages
%
% delete predefined styles
\clearscrheadings
\clearscrplain
%
% What is printed where ...
\IfElseDefined{chapter}{
   \ohead{\pagemark} % header outside: page number
   \ihead{\headmark} % header inside: chapter and section titles
   \ofoot[\pagemark]{} % footer outside: page numbers on plain pages
}{
   \cfoot[\pagemark]{\pagemark} % Mitte unten: Seitenzahlen bei plain
}
% Complete list of possible positions
%\lehead[scrplain-left-even   ]{scrheadings-left-even }
%\cehead[scrplain-center-even ]{scrheadings-center-even }
%\rehead[scrplain-right-even  ]{scrheadings-right-even }
%\lefoot[scrplain-left-even   ]{scrheadings-left-even }
%\cefoot[scrplain-center-even ]{scrheadings-center-even }
%\refoot[scrplain-right-even  ]{scrheadings-right-even }
%\lohead[scrplain-left-odd    ]{scrheadings-left-odd }
%\cohead[scrplain-center-odd  ]{scrheadings-center-odd }
%\rohead[scrplain-right-odd   ]{scrheadings-right-odd }
%\lofoot[scrplain-left-odd    ]{scrheadings-left-odd }
%\cofoot[scrplain-center-odd  ]{scrheadings-center-odd }
%\rofoot[scrplain-right-odd   ]{scrheadings-right-odd }
%\ihead[scrplain-inside       ]{scrheadings-inside }
%\chead[scrplain-centered     ]{scrheadings-centered }
%\ohead[scrplain-outside      ]{scrheadings-outside }
%\ifoot[scrplain-inside       ]{scrheadings-inside }
%\cfoot[scrplain-centered     ]{scrheadings-centered }
%\ofoot[scrplain-outside      ]{scrheadings-outside }

% Shown sections in the header
\IfElseDefined{chapter}{
   \automark[section]{chapter} %[right]{left}
}{
   \automark[subsection]{section} %[right]{left}
}
%
% Lines
\IfDefined{chapter}{%
   % \setheadtopline{} % configures the line above the header
   \setheadsepline{.4pt}[\color{black}] % configures the line below the header
   % \setfootsepline{} % configures the line above the footer
   % \setfootbotline{} % configures the line below the footer
}

%% width of head and foot
\setheadwidth[0pt]{text}
\setfootwidth[0pt]{text}
%   paper % width of paper
%   page  % width of page (paper - BCOR)
%   text  % \textwidth
%   textwithmarginpar % width of text plus margin
%   head  % current width of head
%   foot  % current width of foot

% set chapter pages with heading (or other) style
%\renewcommand*{\chapterpagestyle}{scrheadings}

%\renewcommand*{\partpagestyle}{empty}
%\renewcommand*{\titlepagestyle}{empty}
%\renewcommand*{\indexpagestyle}{empty}

} % end: \IfPackageLoaded{scrpage2}

\EndCodeSection{StyleHeadFoot}
% ~~~~~~~~~~~~~~~~~~~~~~~~~~~~~~~~~~~~~~~~~~~~~~~~~~~~~~~~~~~~~~~~~~~~~~~~
% headings / page opening
% ~~~~~~~~~~~~~~~~~~~~~~~~~~~~~~~~~~~~~~~~~~~~~~~~~~~~~~~~~~~~~~~~~~~~~~~~
\BeginCodeSection{StyleHeadings}

% depth of sections numbering
\setcounter{secnumdepth}{2}
% 0 - chapter
% 1 - section
% 2 - subsection and so on ...

\KOMAoptions{%
%%%% headings
   headings=small  % Small Font Size, thin spacing above and below
   % headings=normal % Medium Font Size, medium spacing above and below
   % headings=big % Big Font Size, large spacing above and below
   %
%%% Add/Dont/Auto Dot behind section numbers 
%%% (see DUDEN as reference)
   % ,numbers=autoenddot
   % ,numbers=enddot
   ,numbers=noenddot
}%

\IfDefined{chapter}{
   \KOMAoptions{%
      headings=noappendixprefix % chapter in appendix as in body text
      % ,headings=nochapterprefix  % no prefix at chapters
      % ,headings=appendixprefix   % inverse of 'noappendixprefix'
      ,headings=chapterprefix    % inverse of 'nochapterprefix'
      % ,headings=openany   % Chapters start at any side
      % ,headings=openleft  % Chapters start at left side
      ,headings=openright % Chapters start at right side      
   }%
}%

% headings left aligned and ragged
\renewcommand*{\raggedsection}{\raggedright} 

\EndCodeSection{StyleHeadings}
% ~~~~~~~~~~~~~~~~~~~~~~~~~~~~~~~~~~~~~~~~~~~~~~~~~~~~~~~~~~~~~~~~~~~~~~~~
% fonts of headings
% ~~~~~~~~~~~~~~~~~~~~~~~~~~~~~~~~~~~~~~~~~~~~~~~~~~~~~~~~~~~~~~~~~~~~~~~~
\BeginCodeSection{StyleHeadingsFonts}


% Default font for sections
\newcommand\SectionFontStyle{\rmfamily}

\IfDefined{chapter}{%
   \setkomafont{chapter}{\Large\bfseries\SectionFontStyle}    % Chapter
}

\setkomafont{sectioning}{\SectionFontStyle}
\setkomafont{section}{\large\bfseries\usekomafont{sectioning}}
\setkomafont{subsection}{\normalsize\bfseries\usekomafont{sectioning}}
\setkomafont{subsubsection}{\normalsize\itshape\usekomafont{sectioning}}
\setkomafont{paragraph}{\rmfamily\itshape} 
\setkomafont{subparagraph}{\rmfamily}

\setkomafont{descriptionlabel}{\itshape}

%\setkomafont{dictum}{}
%\setkomafont{dictumauthor}{}
%\setkomafont{dictumtext}{}
%\setkomafont{disposition}{}
%\setkomafont{footnote}{}
%\setkomafont{footnotelabel}{}
%\setkomafont{footnotereference}{}
%\setkomafont{minisec}{}

\setkomafont{part}{\usekomafont{sectioning}\LARGE}
\setkomafont{partnumber}{\usekomafont{sectioning}\Huge}

\setkomafont{pageheadfoot}{\normalfont\normalcolor\small\rmfamily}
% \setkomafont{pagenumber}{\bfseries\usekomafont{sectioning}}
\setkomafont{pagenumber}{\normalfont\rmfamily\fontshape{b}\selectfont}

%%% --- Titlepage ---
%\setkomafont{subject}{}
%\setkomafont{subtitle}{}
%\setkomafont{title}{}

% colors of headings
\IfDefined{color}{%
  \IfColorDefined{sectioncolor}{%
    \addtokomafont{sectioning}{\color{sectioncolor}}%
    \IfDefined{chapter}{%
      \addtokomafont{chapter}{\color{sectioncolor}}%
    }%
  }{}%
}

\EndCodeSection{StyleHeadingsFonts}
% ~~~~~~~~~~~~~~~~~~~~~~~~~~~~~~~~~~~~~~~~~~~~~~~~~~~~~~~~~~~~~~~~~~~~~~~~
% layout of headings 
% ~~~~~~~~~~~~~~~~~~~~~~~~~~~~~~~~~~~~~~~~~~~~~~~~~~~~~~~~~~~~~~~~~~~~~~~~
\BeginCodeSection{StyleHeadingsLayout}
%%% Remove Space above Chapter.
%%% (NOT recommanded!)
%% Space above Chapter Title
% \renewcommand*{\chapterheadstartvskip}{\vspace{1\baselineskip}}%
%% Space below Chapter Title
% \renewcommand*{\chapterheadendvskip}{\vspace{0.5\baselineskip}}%

%% code taken from 
%% http://tex.stackexchange.com/questions/307522/convert-titlesec-code-to-something-koma-script-like
%% by user >esdd<

% part and chapter
\RedeclareSectionCommand[
  style=chapter,
  beforeskip=-1sp,
  afterskip=1sp,
  innerskip=0pt,
  font=\mdseries\Large,
  prefixfont=\LARGE,
]{part}

\RedeclareSectionCommand[
  innerskip=1pt,
  font=\mdseries\Large,
  prefixfont=\LARGE,
]{chapter}

\renewcommand*{\partformat}{%
  \raisebox{-.5\dp\strutbox}{%
    \makebox[0pt]{%
      \setlength\fboxsep{.5em}%
      \colorbox{white}{%
        \partname\nobreakspace{\Huge\thepart\autodot}%
}}}}

\renewcommand*{\chapterformat}{%
  \mbox{\MakeUppercase{%
    \chapappifchapterprefix{\nobreakspace}}{\Huge\thechapter\autodot}%
    \IfUsePrefixLine{}{\enskip}}%
}

\renewcommand\chapterlineswithprefixformat[3]{%
  \ifstr{#1}{chapter}{%
    #2\nobreak%
    \vspace*{\dimexpr-\ht\strutbox}%
    \rule[-\dp\strutbox]{\textwidth}{.4pt}\\*[.9pc]%
    {\IfColorDefined{sectioncolor}{\color{sectioncolor}}{}#3}%
    \vspace*{\dimexpr-\ht\strutbox-\dp\strutbox+.9pc}\nobreak%
    \rule[-\dp\strutbox]{\textwidth}{.4pt}%
    \par\nobreak%
  }{%
    \ifstr{#1}{part}{%
      \null\vfil
      \fbox{%
        \parbox[t][\dimexpr\height+3\normalbaselineskip][c]
          {\dimexpr\textwidth-2\fboxsep-2\fboxrule\relax}
          {\centering \IfColorDefined{sectioncolor}{\color{sectioncolor}}{}#3}%
      }\nolinebreak%
      \ifnumbered{part}{\hspace*{-.5\textwidth}#2}{}%
      \vfil\newpage\partheademptypage
    }{%
      #2#3%
    }%
  }%
}

% other section levels
\RedeclareSectionCommand[
  beforeskip=-3ex plus -.6ex minus -0.12ex,
  afterskip=2ex plus .05ex
]{section}

\RedeclareSectionCommands[
  beforeskip=-3ex plus -.45ex minus -0.09ex,
  afterskip=2ex plus .05ex
]{subsection,subsubsection}


\EndCodeSection{StyleHeadingsLayout}
% ~~~~~~~~~~~~~~~~~~~~~~~~~~~~~~~~~~~~~~~~~~~~~~~~~~~~~~~~~~~~~~~~~~~~~~~~
% settings and layout of TOC, LOF 
% ~~~~~~~~~~~~~~~~~~~~~~~~~~~~~~~~~~~~~~~~~~~~~~~~~~~~~~~~~~~~~~~~~~~~~~~~
\BeginCodeSection{StyleLayoutTOC}
%%% === Table of Contents ==============================

\setcounter{tocdepth}{3} % Depth of TOC Display

\KOMAoptions{%
   %%% Setting of 'Style' and 'Content' of TOC
   % toc=left, %
   toc=indented,%
}%  

% setup of package titletoc
\IfPackageLoaded{titletoc}{

% default definition of the toc format of sections
%\titlecontents{section}
%[3.8em] % left
%{}  % above code
%{\contentslabel{2.3em}}    % numbered-entry-format
%{\hspace*{-2.3em}}         % numberless-entry-format
%{\titlerule*[1pc]{.}\contentspage} % filler-page-format
%[]      % below code

% Define partial toc for part pages
\IfDefined{part}{
  \newcommand{\PartialToc}[1]{%
    \thispagestyle{plain}
    \startcontents[part]
    \section*{\contentsname}
    \printcontents[part]{}{0}{\setcounter{tocdepth}{#1}}
  }
}

% Thanks to egreg, for providing this code at
% http://tex.stackexchange.com/questions/101773/write-to-back-page-of-part

} % end of \IfPackageLoaded{titletoc}

% Setup using tocstyle
\IfPackageLoaded{tocstyle}{
% predefined styles
% \usetocstyle{standard} % A style similar to the standard classes. 
%                        % \setkomafont has no effect!
\usetocstyle{KOMAlike} % A style similar to the KOMA-Script classes.  
%%            % This is almost the same like standard, but instead 
%%            % of bold face \usekomafont { disposition } will be used if 
%%            % \usekomafont was defined and sans serif, bold face
%%            % (\sffamily\bfseries) if not.
%%            %
%\usetocstyle{classic}  % Like KOMAlike but all page numbers are set 
%%                      % using normal font.
%\usetocstyle{allwithdot}  % Like classic but dots between entry text 
%%                         % and page numbers are used at all depths.
%\usetocstyle{noonewithdot} % Like classic but not dots between entry 
%%                          % text and page numbers are used.
%\usetocstyle{nopagecolumn} % Like noonewithdot but also the gap between 
%                           % text and page numbersis omited.  
}

% \newcommand{\fontTOC}{\sffamily}
\newcommand{\fontTOC}{\rmfamily}


\IfPackageNotLoaded{tocloft}{ % inkompatible
   % apply style of TOC using koma script
   \setkomafont{partentry}{\fontTOC\bfseries\large}
   \setkomafont{partentrypagenumber}{\fontTOC\bfseries}
   \IfElseDefined{chapter}{%
      \setkomafont{chapterentry}{\bfseries\fontTOC}
      \setkomafont{chapterentrypagenumber}{\bfseries\fontTOC}
   }{%
      \setkomafont{sectionentry}{\bfseries\fontTOC}
      \setkomafont{sectionentrypagenumber}{\bfseries\fontTOC}
   }
}


%%% === Appereance of Lists of figures, tables etc.  ===
\KOMAoptions{%
   %%% Setting of 'Style' and 'Content' of Lists 
   %%% (figures, tables etc)
   % --- General List Style ---
   % listof=left, % tabular styles
   listof=indented, % hierarchical style
   % --- Appearance of Lists in TOC
   listof=notoc, % Lists are not part of the TOC
   % listof=totoc, % add Lists to TOC without number
   % listof=totocnumbered, % add Lists to TOC with number
%%% index in toc
   index=nottotoc, % index is not part of the TOC
   % index=totoc, % add index to TOC without number
%%% bib in toc
   % bibliography=nottotoc, % Bibliography is not part of the TOC
   % bibliography=totocnumbered, % add Bibliography to TOC with number
   bibliography=totoc % add Bibliography to TOC without number
}%  

%\IfDefined{chapter}{%
% \KOMAoptions{%
%   % --- chapter highlighting ---
%   % listof=chapterentry, % ??? Chapter starts are marked in figure/table
%   % listof=chaptergapline, % New chapter starts are marked by a gap 
%                            % of a single line
%   listof=chaptergapsmall, % New chapter starts are marked by a gap 
%                            % of a smallsingle line
%   % listof=nochaptergap, % No Gap between chapters
%   %
%   % listof=leveldown, % lists are moved one level down ???
% }
%}

% Subfigures text in List of Figures
\IfPackageLoaded{subfig}{
   \setcounter{lofdepth}{1}  %1 = only figures, 2 = figures and subfigures
}

\EndCodeSection{StyleLayoutTOC}
% ~~~~~~~~~~~~~~~~~~~~~~~~~~~~~~~~~~~~~~~~~~~~~~~~~~~~~~~~~~~~~~~~~~~~~~~~
% pdf packages
% ~~~~~~~~~~~~~~~~~~~~~~~~~~~~~~~~~~~~~~~~~~~~~~~~~~~~~~~~~~~~~~~~~~~~~~~~
\BeginCodeSection{StylePdf}

\IfPackageLoaded{hyperref}{

\hypersetup{
%%% General options
  ,draft=false, % all hypertext options are turned off
  ,final=true   % all hypertext options are turned on
  ,debug=false  % extra diagnostic messages are printed in the log file
  ,hypertexnames=true % use guessable names for links
  ,naturalnames=false % use LaTeX-computed names for links
  ,setpagesize=true   % sets page size by special driver commands
%%% Configuration options
  ,raiselinks=true    % forces commands to reflect the
                      % real height of the link 
  ,breaklinks=true    % Allows link text to break across lines
  ,pageanchor=true    % Determines whether every page is given an implicit
                      % anchor at the top left corner. 
  ,plainpages=false   % Forces page anchors to be named by the arabic
                      % form of the page number, rather than the formatted form.
%%% Extension options
  ,linktocpage=true   % make page number, not text, be link on TOC, LOF and LOT
  ,colorlinks=true    % Colors the text of links and anchors.
}%
\IfColorDefined{pdflinkcolor}{\hypersetup{%
%%% Colors for links
  ,linkcolor  =pdflinkcolor   % Color for normal internal links.
  ,anchorcolor=pdfanchorcolor % Color for anchor text.
  ,citecolor  =pdfcitecolor   % Color for bibliographical citations in text.
  ,filecolor  =pdffilecolor   % Color for URLs which open local files.
  ,menucolor  =pdfmenucolor   % Color for Acrobat menu items.
  ,runcolor   =pdfruncolor    % Color for run links (launch annotations).
  ,urlcolor   =pdfurlcolor    % color magenta Color for linked URLs.
}}{}
\hypersetup{%
%%% PDF-specific display options
  ,bookmarksopen=true     % If Acrobat bookmarks are requested, show them
                          % with all the subtrees expanded.
  ,bookmarksopenlevel=2   % level (\maxdimen) to which bookmarks are open
  ,bookmarksnumbered=true %
  ,bookmarkstype=toc      %
%%% PDF display and information options
  ,pdfpagemode=UseOutlines % Determines how the file is opening in Acrobat:
                          %  UseNone, UseThumbs (show thumbnails),
                          %  UseOutlines (show bookmarks), FullScreen,
                          %  UseOC (PDF 1.5), and UseAttachments (PDF 1.6).
                          %
  ,pdfstartpage=1         % Determines on which page the PDF file is opened.
  ,pdfstartview=FitV      % Set the startup page view
  % options: (same for pdfview, pdfremotestartview)
  %  Fit   Fits the page to the window.
  %  FitH  Fits the width of the page to the window.
  %  FitV  Fits the height of the page to the window.
  %  FitB  Fits the page bounding box to the window.
  %  FitBH Fits the width of the page bounding box to  the window.
  %  FitBV Fits the height of the page bounding box to the window.
  ,pdfremotestartview=Fit % Set the startup page view of remote PDF files
  ,pdfcenterwindow=false  %
  ,pdffitwindow=false     % resize document window to fit document size
  ,pdfnewwindow=false     % make links that open another PDF file 
                          % start a new window
  % options:
  %  SinglePage     Displays a single page; advancing flips the page
  %  OneColumn      Displays the document in one column; continuous scrolling.
  %  TwoColumnLeft  Displays the document in two columns, 
  %                 odd-numbered pages to the left.
  %  TwoColumnRight Displays the document in two columns, 
  %                 odd-numbered pages to the right.
  %  TwoPageLeft    Displays two pages, odd-numbered pages to the left 
  %  TwoPageRight   Displays two pages, odd-numbered pages to the right 
  %
  ,pdfdisplaydoctitle=true  % display document title instead of file name 
} % end: hypersetup

} % end: IfPackageLoaded{hyperref}

\IfPackageLoaded{bookmark}{
   \bookmarksetup{%
   %%% Action options
      ,page=1    % 
      %,view     %
      ,open=true %
      ,openlevel=2 % level to which bookmarks are open
      ,depth=4   % level to which bookmarks are generated
      ,numbered=true
   }%
}

%% disable compression of images in pdf
% \ifpdf
%    \pdfcompresslevel=0
% \fi

% Make figure and not only the number to a link
\IfPackageLoaded{babel}{
  % if babel loaded not necessary
  %\providecommand*{\figurename}{Abbildung}
  %\providecommand*{\tablename}{Tabelle}
  %\providecommand*{\chaptername}{Kapitel}
  % not defined by babel
  \iflanguage{ngerman}{%
    \providecommand*{\secrefname}{Abschnitt}%
    \providecommand*{\eqnrefname}{Gleichung}%
  }{}%
  \iflanguage{english}{%
    \providecommand*{\secrefname}{section}%
    \providecommand*{\eqnrefname}{equation}%
  }{}%
  %
  \IfElsePackageLoaded{hyperref}{
    \newcommand*{\eqnref}[1]{%
        \hyperref[{#1}]{\eqnrefname~(\ref*{#1}})%
    }%
    \newcommand*{\figref}[1]{%
      \hyperref[{#1}]{\figurename~\ref*{#1}}%
    }%
    \newcommand*{\tabref}[1]{%
      \hyperref[{#1}]{\tablename~\ref*{#1}}%
    }%
    \newcommand*{\secref}[1]{%
      \hyperref[{#1}]{\secrefname~\ref*{#1}}%
    }%
    \newcommand*{\chapref}[1]{%
      \hyperref[{#1}]{\chaptername~\ref*{#1}}%
    }%
  }{% hyperref not loaded
    \newcommand*{\eqnref}[1]{%
      \eqnrefname~(\ref*{#1})%
    }%
    \newcommand*{\figref}[1]{%
      \figurename~\ref*{#1}%
    }%
    \newcommand*{\tabref}[1]{%
      \tablename~\ref*{#1}%
    }%
    \newcommand*{\secref}[1]{%
      \secrefname~\ref*{#1}%
    }%
    \newcommand*{\chapref}[1]{%
      \chaptername~\ref*{#1}%
    }%
  }% end: hyperref not loaded
}% \IfPackageLoaded{babel}


\EndCodeSection{StylePdf}
% ~~~~~~~~~~~~~~~~~~~~~~~~~~~~~~~~~~~~~~~~~~~~~~~~~~~~~~~~~~~~~~~~~~~~~~~~
% fix remaining problems
% ~~~~~~~~~~~~~~~~~~~~~~~~~~~~~~~~~~~~~~~~~~~~~~~~~~~~~~~~~~~~~~~~~~~~~~~~
\BeginCodeSection{StyleFixProblems}
% ------------------------------------------------------------------
% Define frontmatter, mainmatter and backmatter if not defined
% because this template shall compile in any koma script class
\makeatletter
\@ifundefined{frontmatter}{%
   \newcommand{\frontmatter}{%
      % (i, ii, iii)
      \pagenumbering{roman}
   }
}{}
\@ifundefined{mainmatter}{%
   % scrpage2 benoetigt den folgenden switch
   % wenn \mainmatter definiert ist.
   \newif\if@mainmatter\@mainmattertrue
   \newcommand{\mainmatter}{%
      %  (1,2,3)
      \pagenumbering{arabic}%
      \setcounter{page}{1}%
   }
}{}
\@ifundefined{backmatter}{%
   \newcommand{\backmatter}{
      % (i, ii, iii)
      \pagenumbering{roman}
   }
}{}
\makeatother

% fix Problem with onlyamsmath active $ char 
% together with the tabu package
% -> switches $ back to its original definition
\IfPackagesLoaded{onlyamsmath,tabu}{%
  \RequirePackage{etoolbox}
  \AtBeginEnvironment{tabu}{\catcode`$=3 } 
}{}
% thanks to egreg for providing this fix.
% The discussion on why this is necessary can be read at
% http://tex.stackexchange.com/questions/35139/restore-original-definition-of

% fix Problem with onlyamsmath active $ char 
% together with the tikz package
% fix incompatibilty problems with tikz and onlyamsmath
\IfPackagesLoaded{onlyamsmath,tikz}{%
	\AtBeginDocument{\catcode`\$=3}
}{}
% thanks to Peter Grill and Christian Feuersänger for providing this fix.
% The discussion on why this is necessary can be read at
% http://tex.stackexchange.com/questions/31860/conflict-onlyamsmath-and-tikz
% http://tex.stackexchange.com/questions/99526/bug-in-pgfplots-or-other-packages



% fix problems with framed and marginnote
\IfPackagesLoaded{marginnote, framed}{%
\ifpdftex{%
 \ifpdfoutput{}{%
     \begingroup
         \makeatletter
         \g@addto@macro\framed{%
            \let\marginnoteleftadjust\FrameSep
            \let\marginnoterightadjust\FrameSep
         }
         \makeatother
     \endgroup
   }% ifpdfoutput
 }{}% ifpdftex
}{}



% ------------------------------------------------------------------
\EndCodeSection{StyleFixProblems}